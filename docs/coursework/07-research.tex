\chapter{\label{research}Исследовательский раздел}

В данном разделе будут описаны технические характеристики устройства, на котором проводились исследования и будет проведено исследование среднего времени ответа на запросы от числа запросов в секунду с использованием кеша приложения и без.

\section{Технические характеристики устройства}

Технические характеристики устройства, на котором проводилось исследование: 

\begin{itemize}[label=--]
	\item операционная система: Майкрософт Windows 10 Домашняя для одного языка \cite{info_ms_windows_home};
	\item оперативная память: 16 Гб;
	\item процессор: 11th Gen Intel® Core™ i7-1185G7 @ 3.00 ГГц × 8.
\end{itemize}

Исследование проводилось на ноутбуке, не подключенном к сети электропитания. В процессе проведение исследования были запущены Docker-контейнеры с базой данных и приложением. Также был запущена среда разработки GoLand, через которую запускались скрипты для проведения исследования.

\section{Проведение первого исследования}

В данном подразделе будет проведено нагрузочное тестирование созданного приложения и будет исследована зависимость среднего времени ответа на запросы от числа запросов в секунду с использованием кеша приложения и без.

Для проведения нагрузочного тестирования был использован инструмент для нагрузочного тестирования Locust \cite{info_locust}.


\subsection{Цель первого исследование}

Целью исследования является проведение сравнительного анализа зависимости среднего времени ответа на запросы от числа запросов в секунду с использованием кеша приложения и без.

\clearpage

\subsection{Наборы варьируемых и фиксированных параметров}


В качестве фиксированных параметров были выбраны следующие:

\begin{itemize}[label=--]
	\item максимальное число пользователей (равно 500);
	\item прирост пользователей в секунду (равно 10);
	\item длительность нагрузочного тестирования (равно 1 минуте);
\end{itemize}

Для проведения исследования были отобраны запросы на выполнение четырех самых основных действий в рамках созданной системы: 

\begin{itemize} 
	\item запрос на получение информации о товаре;
	\item запрос на получение страницы с товарами из каталога товаров;
	\item запрос на выполнение сравнения цен на выбранный товар;
	\item запрос на получение сертификатов соответствия товара.
\end{itemize}

В качестве кеша приложения использовалось in-memory хранилище данных Redis \cite{info_redis}.

\subsection{Результаты первого исследования}
Результаты замеров среднего времени ответа на запросы от числа запросов в секунду с использованием кеширования и без использования кеширования представлены в таблицах  \ref{tbl:cmpResponseTimeByRequests_s_withoutCache} и \ref{tbl:cmpResponseTimeByRequests_s_withCache} на рисунке \ref{img:research}.

\begin{table}[ht]
	\begin{center}
		\begin{threeparttable}
			\caption{Результаты замеров среднего времени ответа на запрос от числа запросов в секунду без использования кеширования}
			\label{tbl:cmpResponseTimeByRequests_s_withoutCache}
			\begin{tabular}{|r|r|}
				\hline
				\bfseries \makecell{Число запросов в секунду} & \bfseries \makecell{Среднее время ответа, мс}  \\
				\hline
				0 & 25.846 \\ 
				\hline
				78 & 50.045 \\ 
				\hline
				275 & 73.115 \\ 
				\hline
				327 & 97.479 \\ 
				\hline
				347 & 121.171 \\ 
				\hline
				362 & 144.350 \\ 
				\hline
				399 & 167.822 \\ 
				\hline
				406 & 187.674 \\ 
				\hline
				420 & 209.065 \\ 
				\hline
				425 & 228.758 \\ 
				\hline
				443 & 250.598 \\ 
				\hline
				450 & 269.343 \\ 
				\hline
				469 & 289.805 \\ 
				\hline
				473 & 309.270 \\ 
				\hline
				470 & 334.602 \\ 
				\hline
				453 & 355.607 \\ 
				\hline
				455 & 377.230 \\ 
				\hline
				442 & 398.857 \\ 
				\hline
				448 & 417.814 \\ 
				\hline
				452 & 436.418 \\ 
				\hline
				453 & 461.885 \\ 
				\hline
				437 & 477.444 \\ 
				\hline
				442 & 501.625 \\ 
				\hline
				431 & 517.916 \\ 
				\hline
				422 & 534.373 \\ 
				\hline
				435 & 558.219 \\ 
				\hline
				439 & 584.407 \\ 
				\hline
				442 & 607.249 \\ 
				\hline
			\end{tabular}
		\end{threeparttable}
	\end{center}
\end{table}

\begin{table}[ht]
	\begin{center}
		\begin{threeparttable}
			\caption{Результаты замеров среднего времени ответа на запрос от числа запросов в секунду  с использованием кеширования}
			\label{tbl:cmpResponseTimeByRequests_s_withCache}
			\begin{tabular}{|r|r|}
				\hline
				\bfseries \makecell{Число запросов в секунду} & \bfseries \makecell{Среднее время ответа, мс}  \\
				\hline
				0 & 28.747 \\ 
				\hline
				108 & 58.885 \\ 
				\hline
				947 & 93.528 \\ 
				\hline
				1007 & 130.303 \\ 
				\hline
				1035 & 166.300 \\ 
				\hline
				1057 & 200.663 \\ 
				\hline
				1141 & 235.700 \\ 
				\hline
				1127 & 260.495 \\ 
				\hline
				1138 & 275.387 \\ 
				\hline
				1145 & 286.911 \\ 
				\hline
				1146 & 296.106 \\ 
				\hline
				1154 & 304.076 \\ 
				\hline
				1164 & 311.796 \\ 
				\hline
				1165 & 317.920 \\ 
				\hline
				1165 & 322.696 \\ 
				\hline
				1161 & 327.387 \\ 
				\hline
				1149 & 330.579 \\ 
				\hline
				1152 & 334.151 \\ 
				\hline
				1158 & 337.071 \\ 
				\hline
				1152 & 340.013 \\ 
				\hline
				1164 & 342.504 \\ 
				\hline
				1163 & 344.504 \\ 
				\hline
				1153 & 346.230 \\ 
				\hline
				1158 & 347.954 \\ 
				\hline
				1165 & 349.993 \\ 
				\hline
				1153 & 351.469 \\ 
				\hline
				1153 & 352.436 \\ 
				\hline
				1165 & 354.096 \\ 
				\hline
			\end{tabular}
		\end{threeparttable}
	\end{center}
\end{table}
\clearpage

\includesvgimage
{research} % Имя файла без расширения (файл должен быть расположен в директории inc/img/)
{f} % Обтекание (без обтекания)
{h} % Положение рисунка (см. figure из пакета float)
{1\textwidth} % Ширина рисунка
{Графики зависимости среднего времени ответа от числа запросов в секунду с использованием кеширования и без} % Подпись рисунка

Анализируя выборки, представленные в таблицах \ref{tbl:cmpResponseTimeByRequests_s_withoutCache} и \ref{tbl:cmpResponseTimeByRequests_s_withCache}, можно увидеть, что при числе запросов, меньших 275, среднее время ответа при использовании кеширования практически не отличается от среднего времени ответа без использования кеширования (из таблицы \ref{tbl:cmpResponseTimeByRequests_s_withoutCache} при 78 запросах в секунду, время ответа составляет $\approx$ 50 мс, а из таблицы \ref{tbl:cmpResponseTimeByRequests_s_withCache} при 108 запросах $\approx$ 58 мс). Это обусловлено тем, что в начале кеш был пуст и первые запросы выполнялись именно к базе данных.

При числе запросов, больших 275, можно увидеть резкий рост среднего времени ответа без использования кеширования (при 327 запроса в секунду среднее время ответа 97.479 мс, а при 442 запросах -- 607.249, что примерно в 6.23 раза больше). Число обрабатываемых запросов в секунду также ограничивается сверху числом 473.

Использование кеша позволило обрабатывать значительно больше запросов в секунду (максимальное число обрабатываемых запросов в секунду равно 1165, что в $\approx$ 2.46 раза больше, чем без использования кеша) и значительно сократило среднее время ответа (максимальное среднее время ответа без кеша, равное 607.249 мс, достигалось мне обработке 442 запросов в секунду, а максимальное время ответа с кешем, равное 354.096 мс, достигалось при обработке 1165 запросов в секунду, что меньше $\approx$ 1.71 раза). Само среднее время ответа при использовании кеша также выросло незначительно (при  обработке 947 запросов в секунду среднее время ответа составляет 93.528, а при 1165 запросах в секунду среднее время ответа составляет 354.096, что примерно  в 3.78 раз больше).

\section{Проведение второго исследования}

В данном подразделе будет проведено исследование зависимости времени выполнения запроса от наличия или отсутствия индексов при различном числе записей.

\subsection{Цель второго исследование}

Целью исследования является проведение сравнительного анализа зависимости времени выполнения запроса от наличия или отсутствия индекса при различном числе записей.

\subsection{Наборы варьируемых и фиксированных параметров}

Замеры времени проводились для числа записей от 10 до 1000. Для каждого числа записей для случая с индексом и без индекса выполнялось 10 запросов, после чего вычислялось среднее арифметическое и отмечалось в качестве результата.

На листинге \ref{lst:create-index-certificate-by-product-id.txt} представлена команда создания индекса для таблицы certificate\_compliance.

\includelisting
{create-index-certificate-by-product-id.txt} % Имя файла с расширением (файл должен быть расположен в директории inc/lst/)
{Команда создания индекса} % Подпись листинга

\clearpage

Для проведения исследования был выбран запрос на получение всех сертификатов товара по его идентификатору в таблице сertificate\_compliance.

\includelisting
{query-get-certificates-by-product-id.txt} % Имя файла с расширением (файл должен быть расположен в директории inc/lst/)
{Запрос на получение сертификатов соответтствия товара} % Подпись листинга

Вместо $\$1$ подставляется идентификатор товара, сертификаты соответствия которого необходимо получить.

\subsection{Результаты второго исследования}

Результаты замеров времени выполнения запроса от наличия или отсутствия индекса при различном числе записей \ref{tbl:cmpRequestTimeByIndex} и на рисунке \ref{img:index-research1}.


\begin{table}[ht]
	\begin{center}
		\begin{threeparttable}
			\caption{Результаты замеров времени выполнения запроса от наличия или отсутствия индекса при различном числе записей}
			\label{tbl:cmpRequestTimeByIndex}
			\begin{tabular}{|r|r|r|}
				\hline
				\bfseries \makecell{Число записей} & \bfseries \makecell{Без индекса, мкс} & \bfseries \makecell{С индексом, мкс}\\
				\hline
				10 & 25.9 & 25.6 \\
				\hline
				100 & 37.7 & 26.1 \\
				\hline
				200 & 52.0 & 26.6 \\
				\hline
				300 & 60.1 & 36.8 \\
				\hline
				400 & 87.7 & 37.6 \\
				\hline
				500 & 101.7 & 38.3 \\
				\hline
				600 & 129.4 & 45.5 \\
				\hline
				700 & 134.3 & 46.8 \\
				\hline
				800 & 159.5 & 51.3 \\
				\hline
				900 & 160.7 & 59.0 \\
				\hline
				1000 & 165.8 & 59.9 \\
				\hline
			\end{tabular}
		\end{threeparttable}
	\end{center}
\end{table}

\clearpage

\includesvgimage
{index-research1} % Имя файла без расширения (файл должен быть расположен в директории inc/img/)
{f} % Обтекание (без обтекания)
{h} % Положение рисунка (см. figure из пакета float)
{1\textwidth} % Ширина рисунка
{Графики зависимости времени выполнения запроса при наличии и отсутствии индекса} % Подпись рисунка

Анализируя полученную в результате исследования выборку, представленную в таблице \ref{tbl:cmpRequestTimeByIndex}, можно заметить, что использование индекса с ростом числа записей позволило сократить выполнение запроса практически в 3 раза (при 10 записях в таблице разница во времени выполнения была всего 0.3 мкс, в то время как при 1000 записей в таблице использование индекса сокращает время выполнения $\approx$ 2.77 раза).

\clearpage

\section*{Вывод}

В данном разделе были описаны технические характеристики устройства, на котором проводились исследование и было проведено исследование среднего времени ответа на запросы от числа запросов в секунду с использованием кеша приложения и без.

В результате первого исследования было выяснено, что использование кеша приложения позволяет значительно увеличить число обрабатываемых запросов в секунду (с использованием кеша максимальное число обрабатываемых запросов в секунду увеличилось с 473 до 1165, что в $\approx$ 2.46 раза больше, чем без использования кеша) а также сократить время ответа на запросы практически в 2 раза (максимальное среднее время ответа без кеша равно 607.249, а максимальное время ответа с кешем равно 354.096 мс, что меньше $\approx$ 1.71 раза).

В результате второго исследования выяснилось, что использование индекса значительно сокращает время выполнения запроса (при 10 записях в таблице разница во времени выполнения была всего 0.3 мкс, в то время как при 1000 записей в таблице запрос с использованием индекса выполняется быстрее $\approx$ в 2.77 раза).




