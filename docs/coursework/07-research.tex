\chapter{Исследовательский раздел}

% проверено мной: пойдет

В разделе представлены результаты проведенных исследований, в котором сравниваются временные характеристики работы реализованного
программного обеспечения.

\section{Технические характеристики}

Технические характеристики устройства, на котором проводились исследования: 

\begin{itemize}[label=--]
	\item операционная система: Ubuntu 22.04.3 LTS x86\_64 \cite{info_os};
	\item оперативная память: 16 Гб;
	\item процессор: 11th Gen Intel® Core™ i7-1185G7 @ 3.00 ГГц × 8.
\end{itemize}

Исследования проводились на ноутбуке, который был подключен к сети электропитания. Во время проведения исследований ноутбук был загружен только встроенными приложениями окружения и самим окружением.

\section{Проведение первого исследования}

\subsection{Цель первого исследования}

Целью первого исследования является проведение сравнительного анализа времени построения одного кадра ландшафта в зависимости от его линейных размеров, заданных в количестве треугольных полигонов. 

\clearpage

\subsection{Наборы варьируемых и фиксированных параметров}

Замеры проводились для ландшафта размером $10\times10$, $50\times50$, $100\times100$, $200\times200$, $300\times300$, $400\times400$, $500\times500$, $600\times600$, $700\times700$.

Для каждого размера ландшафта измерялось время 20 визуализаций, после чего вычислялось среднее арифметическое из полученных значений.

В качестве фиксированных параметров были выбраны следующие:

\begin{itemize}[label=--]
	\item состояние (равно 25)
	\item частота (равна 1)
	\item амплитуда (равна 1)
	\item лакунарность (равна 4)
	\item стойкость (равна 0.25)
	\item число октав (равно 8)
	\item максимальная высота (равно 1000)
	\item уровень моря (равен 0)
	\item позиция источника освещения (равна точке с координатами (0, 0, 600))
	\item коэффициент диффузного отражения источника (равна 0.5)
	\item интенсивность источника (равна 3)
\end{itemize}

\clearpage

\subsection{Результаты первого исследования}

Результаты замеров времени построения одного кадра ландшафта от размеров ландшафта приведены в таблице \ref{tbl:RenderLandscapeTimeBySize} и на рисунке \ref{img:study1}.

\begin{table}[ht]
	\begin{center}
		\begin{threeparttable}
			\caption{Результаты замеров времени построения одного кадра ландшафта от линейных размеров ландшафта}
			\label{tbl:RenderLandscapeTimeBySize}
			\begin{tabular}{|c|r|}
				\hline
				\bfseries \makecell{Размеры ландшафта, \\ количество \\ треугольных  полигонов} & \bfseries \makecell{Время построения, \\ секунды}  \\
				\hline
				$10\times10$ &  0.035 \\
				\hline
				$50\times50$ &  0.106 \\
				\hline
				$100\times100$ &  0.340 \\
				\hline
				$200\times200$ &  1.134 \\
				\hline
				$300\times300$ &  2.595 \\
				\hline
				$400\times400$ &  4.899 \\
				\hline
				$500\times500$ &  8.524 \\
				\hline
				$600\times600$ &  12.226 \\
				\hline
				$700\times700$ &  17.154 \\
				\hline
			\end{tabular}
		\end{threeparttable}
	\end{center}
\end{table}

\clearpage

\includesvgimage
{study1} % Имя файла без расширения (файл должен быть расположен в директории inc/img/)
{f} % Обтекание (без обтекания)
{h} % Положение рисунка (см. figure из пакета float)
{1\textwidth} % Ширина рисунка
{Результаты замеров времени построения одного кадра ландшафта от размеров ландшафта} % Подпись рисунка
	

С помощью метода наименьших квадратов было установлено, что полученные дискретные значения результатов аппроксимируются квадратичной функцией вида $a \cdot x^2 + b \cdot x + c$ с невязкой, имеющей значение $\approx 0.135$.

Таким образом, по результатам первого исследования можно сделать вывод о том, что в зависимости от размеров ландшафта наблюдается квадратичная зависимость времени построения кадра. 
Это означает, что увеличение размеров ландшафта приводит к более значительному увеличению времени построения кадра.


\clearpage

\section{Проведение второго исследования}

\subsection{Цель второго исследования}

Целью второго исследования является проведение сравнительного анализа времени генерации одного кадра ландшафта в зависимости от количества октав. 

\subsection{Наборы варьируемых и фиксированных параметров}

Замеры проводились для ландшафта с количеством октав, равным 1, 2, 3, 4, 5, 6, 7, 8, 9.

Для каждой октавы измерялось время 20 визуализаций, после чего вычислялось среднее арифметическое из полученных значений.

В качестве фиксированных параметров были выбраны следующие:

\begin{itemize}[label=--]
	\item состояние (равно 25)
	\item частота (равна 1)
	\item амплитуда (равна 1)
	\item лакунарность (равна 4)
	\item стойкость (равна 0.25)
	\item размеры ландшафта (равны $120\times120$)
	\item максимальная высота (равно 1000)
	\item уровень моря (равен 0)
	\item позиция источника освещения (равна точке с координатами (0, 0, 600))
	\item коэффициент диффузного отражения источника (равна 0.5)
	\item интенсивность источника (равна 3)
\end{itemize}


\subsection{Результаты второго исследования}

Результаты замеров времени построения одного кадра ландшафта от его размеров приведены в таблице \ref{tbl:RenderLandscapeTimeByOctaves} и на рисунке \ref{img:study2}.

\begin{table}[ht]
	\begin{center}
		\begin{threeparttable}
			\caption{Результаты замеров времени построения одного кадра ландшафта от количества октав}
			\label{tbl:RenderLandscapeTimeByOctaves}
			\begin{tabular}{|c|r|}
				\hline
				\bfseries \makecell{Число октав} & \bfseries \makecell{Время построения, секунды}  \\
				\hline
				1 &  0.346 \\
				\hline
				2 &  0.452 \\
				\hline
				3 &  0.565 \\
				\hline
				4 &  0.690 \\
				\hline
				5 &  0.862 \\
				\hline
				6 &  1.040 \\
				\hline
				7 &  1.199 \\
				\hline
				8 &  1.392 \\
				\hline
				9 &  1.538 \\
				\hline
			\end{tabular}
		\end{threeparttable}
	\end{center}
\end{table}

\clearpage

\includesvgimage
{study2} % Имя файла без расширения (файл должен быть расположен в директории inc/img/)
{f} % Обтекание (без обтекания)
{h} % Положение рисунка (см. figure из пакета float)
{1\textwidth} % Ширина рисунка
{Результаты замеров времени построения одного кадра ландшафта от количества октав} % Подпись рисунка

С помощью метода наименьших квадратов было установлено, что полученные дискретные значения результатов аппроксимируются линейной функцией вида $a \cdot x + b$ с невязкой, имеющей значение $\approx 0.01$.

С ростом числа октав наблюдается линейный рост времени построения одного кадра: для построения одного кадра ландшафта с 9-ю октавами требуется в 4.44 раза больше времени, чем  для построения одного кадра ландшафта с 1-й октавой.

\clearpage

\section{Вывод}

В данном разделе были проведены исследования по замеру времени построения одного кадра ландшафта от размеров ландшафта и времени построения одного кадра  ландшафта от числа октав.

Время построения одного кадра ландшфта в зависимости от размеров ландшафта имеет квадратичную зависимость и начинает резко увеличиваться начиная с размеров ландшафта в $200\times200$: для отображения ландшафта такой размерности по сравнению с размером  $100\times100$ требуется в 3.33 раза больше времени, для отображения ландшафта размером $700\times700$ по сравнению с размером  $300\times300$ требуется в 6.61 раза больше времени.

Время построения одного кадра ландшафта в зависимости от числа октав имеет линейную зависимость, что подтверждает ожидаемые результаты: реализация алгоритма шума Перлина с применением нескольких октав состоит из одного цикла по числу октав, где на каждой итерации цикла генерируется шум Перлина для текущих частоты и амплитуды шума.



