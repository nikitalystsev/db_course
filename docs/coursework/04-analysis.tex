\chapter{Аналитический раздел}

В данном разделе ... % дописать

\section{Анализ предметной области}

\subsection{Росстат}

\textbf{Росстат} ---  федеральный орган исполнительной власти, осуществляющий функции по формированию официальной статистической информации о социальных, экономических, демографических, экологических и других общественных процессах в Российской Федерации. \cite{info_rosstat}.

Одним из направлений сбора статистики являются \texttt{цены, инфляция}. Организации и их филиалы, индивидуальные предприниматели предоставляют в Росстат статистические отчеты, на основе которых производится расчет цен на различного рода товары и услуги.

\subsection{Элементы продуктовой корзины первой необходимости}

\textbf{Потребительская корзина (ПК)} --- это минимальный набор продуктов питания и непродовольственных товаров и услуг, необходимых для сохранения здоровья человека и обеспечения его жизнедеятельности\cite{info_consumer_basket2}. ПК помогает вести статистику в масштабах страны: сколько население должно потреблять и каковы реальные объёмы потребления.

ПК формируется отдельно для детей, трудоспособного населения и пенсионеров. Росстат собирает и обрабатывает информацию о фактических основных расходах в малоимущих семьях, а затем соотносит с рекомендациями учёных по минимальным потребностям человека.

Потребительская корзина состоит из 3-х частей:

\begin{itemize}[label=--]
	\item продукты питания;
	\item непродовольственные товары;
	\item услуги.
\end{itemize}

\clearpage

К продуктам питания следующие 11 групп товаров \cite{info_consumer_basket2}:

\begin{itemize}[label=--]
	\item хлебные продукты (хлеб и макаронные изделия в пересчете на муку, мука, крупы, бобовые);
	\item картофель;
	\item овощи и бахчевые;
	\item фрукты свежие;
	\item cахар и кондитерские изделия в пересчете на сахар;
	\item мясопродукты;
	\item рыбопродукты;
	\item молоко и молокопродукты в пересчете на молоко;
	\item яйца;
	\item масло растительное, маргарин и другие жиры;
	\item прочие продукты (соль, чай, специи).
\end{itemize}

\subsection{ФГИС <<Меркурий>>}

\textbf{ФГИС <<Меркурий>>} --- автоматизированная система, предназначенная для отслеживания продуктов питания на всей цепи производства и перемещения до точки реализации.

Работа с <<Меркурием>> заключается в создании и <<гашении>> \texttt{ветеринарно-сопроводительных документов (ВСД)} на всех этапах движения товара: от производства и переработки до продажи или утилизации.

\clearpage

Создание ФГИС <<Меркурий>> позволило достичь следующих целей \cite{info_mercury, info_mercury2}:

\begin{itemize}[label=--]
	\item защита потребителя от некачественной и небезопасной продукции, а все население страны от экономических и социальных угроз;
	\item обеспечение прозрачности и эффективности действий надзорных органов в борьбе с мошенничеством;
	\item минимизация бюрократии и предоставление удобного прозрачного механизма для комфортной работы частного бизнеса.
\end{itemize}

\section{Существующие решения}

На рынке существует большое количество сервисов для мониторинга цен на продукты в различных магазинах. Наиболее популярными являются:

\begin{itemize}[label=--]
	\item \texttt{<<Едадил>>};
	\item \texttt{SkidkaOnline};
	\item \texttt{Price.ru}.
\end{itemize}

Критерии, по которым будет произведено сравнение:

\begin{itemize}[label=--]
	\item \textbf{K1} --- возможность цены на конкретный товар в разных магазинах;
	\item \textbf{K2} --- возможность оставить отзыв о товаре;
	\item \textbf{K3} --- покрытие магазинов;
	\item \textbf{K4} --- наличие информации об акциях на товары;
	\item \textbf{K5} --- возможность просмотра динамики изменения цены;
\end{itemize}

\clearpage

\begin{table}[ht]
	\small
	\begin{center}
		\begin{threeparttable}
			\caption{Сравнение существующих решений}
			\label{tbl:exist_sol}
			\begin{tabular}{|c|c|c|c|c|c|}
				\hline
				Решения & \textbf{K1} & \textbf{K2} & \textbf{K3} & \textbf{K4} & \textbf{K5} \\
				\hline
				\texttt{<<Едадил>>} & + & + & Высокое & + & - \\
				\hline
				\texttt{SkidkaOnline} & - & + & Высокое & + & - \\
				\hline
				\texttt{Price.ru} & + & + & Низкое & + & + \\
				\hline
			\end{tabular}
		\end{threeparttable}			
	\end{center}
\end{table}

\section{Формализация задачи}


%Сцена состоит из следующих объектов:
%
%\begin{itemize}[label=--]
%	\item ландшафт --- трехмерная модель, описываемая полигональной сеткой \cite{info_landscape};
%	\item источник света --- материальная точка, испускающая лучи света.
%\end{itemize}
%
%\section{Анализ способов представления данных о ландшафте}
%
%Существует несколько основных принципов представления данных для хранения информации о ландшафте \cite{info_dataLandscapePresent}:
%
%\begin{itemize}[label=--]
%	\item регулярная сетка высот (карта высот);
%	\item иррегулярная сетка вершин и связей, их соединяющих;
%	\item посегментная карта высот.
%\end{itemize}
%
%\subsection{Регулярная сетка}
%
%Данные представлены в виде двумерного массива \cite{info_heightMap}. 
%Каждый его элемент имеет свои индексы $[i, j]$, являющиеся координатами расположения точки на плоскости. 
%Для каждой вершины с индексами $[i, j]$ в двумерном массиве определяется соответствующее значение высоты $h_{ij}$. 
%На рисунке \ref{img:analysis_heightMap} показан пример представления данных о ландшафте с помощью карты высот.
%
%\clearpage
%
%\includeimage
%{analysis_heightMap} % Имя файла без расширения (файл должен быть расположен в директории inc/img/)
%{f} % Обтекание (без обтекания)
%{h} % Положение рисунка (см. figure из пакета float)
%{1\textwidth} % Ширина рисунка
%{Пример представления данных о ландшафте с помощью карты высот \cite{info_heightMap}} % Подпись рисунка
%
%К преимуществам данного способа можно отнести наглядность и простоту изменения данных, легкость нахождения координат и высоты на карте, возможность более точно производить динамическое освещение. 
%Однако, у такого метода есть один существенный недостаток --- избыточность данных (например, при задании плоскости) \cite{info_dataLandscapePresent}.
%
%\subsection{Иррегулярная сетка}
%
%Иррегулярная сетка позволяет размещать точки или узлы в произвольных местах в зависимости от необходимости и особенностей ландшафта.
%
%У такого метода есть несколько существенных недостатков \cite{info_dataLandscapePresent}:
%
%\begin{itemize}[label=--]
%	\item многие алгоритмы построения ландшафтом предназначены для регулярных сеток высот, поэтому оптимизация этих алгоритмов под иррегулярную сетку потребует значительных усилий и времени;
%	\item из-за неравномерного расположения вершинных точек друг к другу возникает сложность при создании динамического освещения;
%	\item сложности при хранении, модификации и просмотре такого ландшафта.
%\end{itemize}
%
%К плюсам данного способа можно отнести использование меньшей информации для построения ландшафта \cite{info_dataLandscapePresent}. 
%
%\subsection{Посегментная карта высот}
%
%Суть этого метода заключается в разделении изначальной области на
%небольшие участки (например, квадратной формы), и генерации высот на
%каждом участке отдельно \cite{info_dataLandscapeSegment}.
%
%К преимуществам данного способа можно отнести возможность представления больших открытых пространств, возможность хранения информации о других объектах (строения, пещеры, скалы), возможность создания нескольких вариантов одного и того же сегмента, но с разным уровнем детализации. 
%К недостаткам можно отнести проблему стыковки разных сегментов, неочевидность представления и проблему модификации данных \cite{info_dataLandscapePresent}.
%
%\section{Анализ алгоритмов процедурной генерации ландшафта}
%
%Главным критерием выбора алгоритма будет качество получаемого ландшафта, поскольку для достижения цели необходимо создавать правдоподобный рельеф.
%
%\subsection{Алгоритм Diamond-Square}
%
%Данный алгоритм является расширением одномерного алгоритма $midpoint \hspace{0.25cm} displacement$ \cite{info_midpDipl} на двумерную плоскость. Алгоритм $Diamond-Square$ \cite{info_diaSqu} выполняется рекурсивно и начинает работу с двумерного массива размера $2^n + 1$. 
%В четырёх угловых точках массива устанавливаются начальные значения высот. Шаги $diamond$ и $square$ выполняются поочередно до тех пор, пока все значения массива не будут установлены.
%
%\begin{enumerate}[label={\arabic*)}]
%    \item Шаг $diamond$ -- для каждого квадрата в массиве, устанавливается срединная точка, которой присваивается среднее арифметическое из четырёх угловых точек плюс случайное значение.
%	\item Шаг $square$ -- берутся средние точки граней тех же квадратов, в которые устанавливается среднее значение от четырёх соседних с ними по осям точек плюс случайное значение.
%\end{enumerate}
%
%На рисунке \ref{img:analysis_diaSqu} показаны последовательность действий алгоритма $Diamond-Square$ на примере массива 5x5.
%
%\includeimage
%{analysis_diaSqu} % Имя файла без расширения (файл должен быть расположен в директории inc/img/)
%{f} % Обтекание (без обтекания)
%{h} % Положение рисунка (см. figure из пакета float)
%{1\textwidth} % Ширина рисунка
%{Шаги, проходимые алгоритмом Diamond-Square на примере массива 5х5 \cite{info_diaSquWiki}} % Подпись рисунка
%
%К преимуществам данного алгоритма можно отнести простоту его реализации и возможность генерации различных ландшафтов с разнообразной детализацией.
%
%К недостаткам можно отнести создание вертикальных и горизонтальных <<складок>> на краях карты из-за наиболее значительного возмущения, происходящего в прямоугольной сетке \cite{info_diaSquWiki}, а также сложности с контролем получаемого ландшафта.
%
%\subsection{Холмовой алгоритм}
%
%Это простой итерационный алгоритм, основанный на нескольких входных параметрах. Алгоритм изложен в следующих шагах \cite{info_hillAlg}:
%
%\begin{enumerate}[label={\arabic*)}]
%	\item создается двухмерный массив и инициализируется нулевым уровнем;
%	\item берется случайная точка на ландшафте или около его границ (за границами) и случайный радиус в заранее заданных пределах. Выбор этих пределов влияет на вид ландшафта -- либо он будет пологим, либо скалистым; 
%	\item в выбранной точке <<поднимается>> холм заданного радиуса;
%	\item выполнение пунктов 1) - 3) $n$ раз, где $n$ -- выбранное количество шагов.
%	\item проводится нормализация ландшафта;
%	\item проводится <<долинизация>> ландшафта.
%\end{enumerate}
%
%К преимуществам данного алгоритма можно отнести простоту его реализации, а также возможность контроля <<гористости>> ландшафта за счет этапов нормализации и долинизации.
%
%Данный алгоритм имеет несколько недостатков:
%
%\begin{itemize}[label=--]
%	\item c помощью этого алгоритма тяжело смоделировать склон холма, или
%	горы, так как в процессе генерации ландшафта используются только
%	гладкие полусферы \cite{info_hillAlgFlaws};
%	\item этот алгоритм неприменим, когда требуется детализировано
%	отобразить лишь часть всего ландшафта \cite{info_hillAlgFlaws}.
%\end{itemize}
%
%\subsection{Алгоритм шума Перлина}
%
%Это математический алгоритм по генерированию процедурной текстуры псевдо-случайным методом \cite{info_perlinNoiseWiki}. 
%Этот алгоритм может быть реализован для n-мерного пространства, но чаще его используют для одно-, двух-, трехмерного случая.
%
%Рассмотрим версию этого алгоритма для двумерного случая:
%
%Для карты высот создается сетка точек и в каждой точке сетки генерируется псевдослучайный единичный вектор-направления градиента. 
%Для каждой точки с координатами $(x, y)$ из карты высот определяется, в какой ячейке сетки находится точка, генерируются вектора, идущие от точек ячейки сетки до этой точки.
%На рисунке \ref{img:analysis_perlinNoise} показан пример векторов для точки $(x, y)$.
%
%\includeimage
%{analysis_perlinNoise} % Имя файла без расширения (файл должен быть расположен в директории inc/img/)
%{f} % Обтекание (без обтекания)
%{h} % Положение рисунка (см. figure из пакета float)
%{0.35\textwidth} % Ширина рисунка
%{Пример сгенерированных векторов для точки $(x, y)$} % Подпись рисунка
%
%Вычисляются четыре скалярных произведения векторов-направлений
%градиента и вектора, идущего от связанной с ним точки сетки к точке с
%координатами $(x, y)$.
%
%Далее, чтобы получить значение высоты $z$ в точке $(x, y)$, необходимо провести двумерную интерполяцию на основе полученных скалярных произведений. Для создания плавного перехода между значениями предварительно производятся вычисления весов измерений по $x$ и по $y$. Веса определяются функцией
%
%\begin{equation}
%	\label{equ:perlin}
%	smootherstep(t) = 6t^5 - 15t^4 + 10t^3
%\end{equation}
%	
%где вместо $t$ подставляются значения $x$ и $y$. 
%Далее, используя метод последовательной интерполяции по каждому измерению, получаем значения двух одномерных линейных интерполяций с использованием веса по $x$, и в конце проводим одну одномерную линейную интерполяцию с этими вычисленными значениями, но уже по весу измерения $y$. 
%Полученное значение и будет высотой в данной точке $(x, y)$ карты высот \cite{info_perlinNoise}. 
%
%Чтобы контролировать генерацию шума, существует набор параметров:
%
%\begin{itemize}[label=--]
%	\item число октав --- количество уровней детализации шума;
%	\item лакунарность --- множитель, который определяет изменение частоты с ростом октавы;
%	\item стойкость --- множитель, который определяет изменение амплитуды с ростом октавы.
%\end{itemize}
%
%Для достижения качественного детализированного ландшафта можно смешивать шумы разных частот и амплитуд.
%
%Преимущества:
%
%\begin{itemize}[label=--]
%	\item комбинация различных октав в алгоритме шума Перлина позволяет
%	добавлять дополнительные уровни детализации к сгенерированному
%	шуму, тем самым повышая реалистичность и детализацию 
%	ландшафта;
%	\item зная лишь параметры генерации, можно получить высоту в любой
%	точке карты без необходимости знания высот в соседних точках
%	карты.
%\end{itemize}
%
%Недостатком является низкая детализация и реалистичность сгенерированного ландшафта без использования механизма комбинации октав.
%
%\section{Анализ алгоритмов удаления невидимых линий и поверхностей}
% 
%Главным требованием при выборе алгоритма будут высокая скорость
%работы, чтобы пользователю не приходилось ждать долгой загрузки
%изображения.
%
%\subsection{Алгоритм Робертса}
%
%Алгоритм Робертса работает в объектном пространстве только с
%выпуклыми телами. Если тело не является выпуклым, то его предварительно
%нужно разбить на выпуклые составляющие \cite{info_rodjers}.
%
%Этот алгоритм выполняется в 4 этапа:
%
%\begin{enumerate}[label={\arabic*)}]
%	\item подготовка исходных данных – составление матрицы тела для
%	каждого тела сцены;
%	\item удаление ребер, экранируемых самим телом;
%	\item удаление ребер, экранируемых другими телами;
%	\item удаление линий пересечения тел, экранируемых самими телами и
%	другими телами, связанными отношением протыкания.
%\end{enumerate}
%
%Алгоритм работает только с выпуклыми телами, что является недостатком данного алгоритма. Также к недостатку можно отнести то, что вычислительная сложность теоретически растет как квадрат числа объектов. К преимуществам можно отнести высокую точность вычислений.
%
%\subsection{Алгоритм, использующий Z-буфер}
%
%Это один из простейших алгоритмов удаления невидимых поверхностей. Работает этот алгоритм в пространстве изображения\cite{info_rodjers}. 
%
%В данном алгоритме используется два буфера: буфер кадра и z-буфер. Буфер кадра используется для заполнения атрибутов (интенсивности) каждого пикселя в пространстве изображения. Z-буфер -- отдельный буфер глубины, используемый для запоминания координаты z или глубины каждого видимого пикселя в пространстве изображения \cite{info_rodjers}.
%
%Вначале в z-буфер заносятся минимально возможные значения $z$, а буфер
%кадра заполняется заполняются фоновым значением интенсивности или цвета.
%Затем каждый многоугольник преобразовывается в растровую форму. 
%Суть работы алгоритма заключается в следующем: в процессе работы глубина
%(значение координаты $z$) каждого нового пикселя, который надо занести в буфер кадра сравнивается с глубиной того пикселя, который уже есть в z-буфере. 
%Если новый пиксель оказывается расположен ближе к наблюдателю, то новый пиксель заносится в буфер кадра. При этом в z-буфер заносится глубина нового пикселя.
%Если сравнение дало противоположный результат, то никаких действий не
%производится. 
%
%К преимуществам данного алгоритма относится простота его реализации, возможность работы со сценами любой сложности, отсутствие необходимости в сортировке по приоритету глубины.
%
%К недостаткам можно отнести большой объем требуемой памяти, трудоемкость и высокая стоимость устранения лестничного эффекта, а также реализации эффектов прозрачности и просвечивания \cite{info_rodjers}.
%
%\subsection{Алгоритм обратной трассировки лучей}
%
%Наблюдатель видит объект посредством испускаемого источником света,
%который падает на этот объект и согласно законам оптики некоторым путем
%доходит до глаза наблюдателя. Алгоритм имеет такое название, потому что
%эффективнее с точки зрения вычислений отслеживать пути лучей в обратном
%направлении, то есть от наблюдателя к объекту \cite{info_rodjers}.
%
%Преимуществами данного алгоритма являются возможность его использования в параллельных вычислительных системах и высокая реалистичность получаемого изображения, а недостатком -- большое количество вычислений и медленная работа алгоритма.
%
%\section{Анализ моделей освещения}
%
%В данном разделе будут рассмотрены две модели освещения: модель Ламберта и модель Фонга.
%
%\subsection{Модель освещения Ламберта}
%
%Модель Ламберта моделирует идеальное диффузное освещение. 
%Считается, что свет при попадании на поверхность рассеивается равномерно во все стороны.
%При расчете такого освещения учитывается только ориентация поверхности  (нормаль $N$) и направление на источник света (вектор $L$). 
%Рассеянная составляющая рассчитывается по закону косинусов (закон Ламберта) \cite{info_lightModels}.
%
%\subsection{Модель освещения Фонга}
%
%Модель Фонга – классическая модель освещения. 
%Модель представляет собой комбинацию диффузной составляющей (модели Ламберта) и зеркальной составляющей и работает таким образом, что кроме равномерного освещения на  материале может еще появляться блик. Местонахождение блика на объекте, освещенном по модели Фонга, определяется из закона равенства углов падения и отражения. 
%Если наблюдатель находится вблизи углов отражения, яркость соответствующей точки повышается \cite{info_lightModels}.
%
%\section{Анализ алгоритмов закраски}
%
%Методы закраски используются для затенения полигонов модели в
%условиях некоторой сцены, имеющей источники освещения. 
%Существует три
%основных алгоритма, позволяющих закрасить полигональную модель.
%
%\subsection{Алгоритм простой закраски}
%
%Суть данного алгоритма заключается в том, что для каждой грани объекта
%находится вектор нормали, и с его помощью в соответствии с выбранной
%моделью освещения вычисляется значение интенсивности, с которой
%закрашивается вся грань.
%При данной закраске все плоскости (в том числе и те, что аппроксимируют
%фигуры вращения), будут закрашены однотонно, что в случае с фигурами
%вращения будет давать ложные ребра \cite{info_rodjers}.
%
%К преимуществам можно отнести простоту реализации алгоритма и его быстродействие. 
%В качестве недостатков можно выделить нереалистичность результата.
%
%\subsection{Алгоритм закраски по Гуро}
%
%Данный алгоритм позволяет получить более сглаженное изображение. 
%Это достигается благодаря тому, что разные точки грани закрашиваются разным
%значением интенсивности.
%
%Алгоритм состоит из следующих шагов:
%
%\begin{enumerate}[label={\arabic*)}]
%	\item вычисление векторов нормалей к каждой грани;
%	\item вычисление векторов нормали к каждой вершине грани путем усреднения нормалей примыкающих граней;
%	\item вычисление интенсивности в вершинах грани в соответствии с выбранной моделью освещения;
%	\item выполнение линейной интерполяции интенсивности вдоль ребер;
%	\item выполнение линейной интерполяции вдоль сканирующей строки.
%\end{enumerate}
%
%Преимуществами данного алгоритма являются хорошее сочетание с моделью освещения с диффузным отражением, а также более высокая, по сравнению с алгоритмом простой закраски, реалистичность получаемого изображения.
%
%К недостаткам можно отнести отсутствие учета кривизны поверхности \cite{info_rodjers}.
%
%\subsection{Алгоритм закраски по Фонгу}
%
%В алгоритме закраски по Фонгу используется билинейная интерполяция не интенсивностей в вершинах полигона, а билинейная интерполяция векторов нормалей. 
%Благодаря такому подходу изображение получается более реалистичным. 
%Однако для достижения такого результата требуется больше вычислительных затрат \cite{info_rodjers}.
%
%\clearpage
%
%Данный алгоритм состоит из следующих шагов:
%
%\begin{enumerate}[label={\arabic*)}]
%	\item вычисление векторов нормалей к каждой грани;
%	\item вычисление векторов нормали к каждой вершине грани путем усреднения нормалей примыкающих граней;
%	\item выполнение линейной интерполяции нормалей вдоль ребер;
%	\item выполнение линейной интерполяции нормалей вдоль сканирующей строки;
%	\item вычисление интенсивности в каждой вершине.
%\end{enumerate}
%
%\section*{Вывод}
%
%% по точке выравнивание
%\begin{table}[ht]
%	\small
%	\begin{center}
%		\begin{threeparttable}
%			\caption{Сравнение способов представления данных о ландшафте}
%			\label{tbl:dataLandscapePresent}
%			\begin{tabular}{|c|c|c|}
%				\hline
%				Способ & \makecell{Наглядность \\ представления \\ данных} & \makecell{Сложность \\ модификации \\ данных} \\
%				\hline
%				\makecell{Регулярная сетка} & Высокая & Низкая  \\
%				\hline
%				\makecell{Иррегулярная сетка} & Высокая & Средняя  \\
%				\hline
%				\makecell{Посегментная карта высот} & Средняя & Высокая  \\
%				\hline
%			\end{tabular}
%		\end{threeparttable}			
%	\end{center}
%\end{table}     
%
%
%% по точке выравнивание
%\begin{table}[ht]
%	\small
%	\begin{center}
%		\begin{threeparttable}
%			\caption{Сравнение алгоритмов процедурной генерации ландшафта}
%			\label{tbl:genLandscapeAlgs}
%			\begin{tabular}{|c|c|c|c|}
%				\hline
%				Алгоритм & \makecell{Качество \\ ландшафта} & \makecell{Отстутствие \\ артефактов} & \makecell{Контроль \\ ландшафта} \\
%				\hline
%				Diamond-Square & Среднее & -- & Низкая   \\
%				\hline
%				\makecell{Холмовой \\ алгоритм} & Среднее & + & Средний \\
%				\hline
%				Шум Перлина & Высокое & + & Высокая  \\
%				\hline
%			\end{tabular}
%		\end{threeparttable}			
%	\end{center}
%\end{table}      
%
%\clearpage
%
%% по точке выравнивание
%\begin{table}[ht]
%	\small
%	\begin{center}
%		\begin{threeparttable}
%			\caption{Сравнение алгоритмов удаления невидимых линий и поверхностей}
%			\label{tbl:delInvisibleAlgs}
%			\begin{tabular}{|c|c|c|c|}
%				\hline
%				Алгоритм & \makecell{Сложность \\ алгоритма} & \makecell{Скорость \\ работы} & \makecell{Типы \\ объектов} \\
%				\hline
%				\makecell{Алгоритм Робертса} & $O(n^2)$ & Средняя & \makecell{Выпуклые \\ многогранники}   \\
%				\hline
%				\makecell{Алгоритм с $Z$-буфером} & $O(np)$ & Высокая & Произвольные  \\
%				\hline
%				\makecell{Алгоритм с обратной \\ трассировки лучей} & $O(np)$ & Низкая & Произвольные  \\
%				\hline
%			\end{tabular}
%		\end{threeparttable}			
%	\end{center}
%\end{table}   
%
%% по точке выравнивание
%\begin{table}[ht]
%	\small
%	\begin{center}
%		\begin{threeparttable}
%			\caption{Сравнение моделей освещения}
%			\label{tbl:lightModels}
%			\begin{tabular}{|c|c|c|}
%				\hline
%				Модель освещения & \makecell{Реалистичность \\ изображения} & \makecell{Объем вычислений} \\
%				\hline
%				\makecell{Модель Ламберта} & Средняя & Средний   \\
%				\hline
%				\makecell{Модель Фонга} & Высокая & Большой \\
%				\hline
%			\end{tabular}
%		\end{threeparttable}			
%	\end{center}
%\end{table} 
%
%% по точке выравнивание
%\begin{table}[ht]
%	\small
%	\begin{center}
%		\begin{threeparttable}
%			\caption{Сравнение алгоритмов закраски}
%			\label{tbl:drawAlgs}
%			\begin{tabular}{|c|c|c|c|}
%				\hline
%				Алгоритм закраски & \makecell{Скорость \\ работы} & \makecell{Реалистичность \\ изображения} & \makecell{Сочетание с \\ диффузным отражением} \\
%				\hline
%				\makecell{Простая закраска} & Высокая & Низкая & Высокое  \\
%				\hline
%				\makecell{Закраска по Гуро} & Средняя & Средняя & Высокое \\
%				\hline
%				\makecell{Закраска по Фонгу} & Низкая & Высокая & Cредняя \\
%				\hline
%			\end{tabular}
%		\end{threeparttable}			
%	\end{center}
%\end{table} 
%  
%В данном разделе была проведена формализация объектов синтезируемой сцены, рассмотрены существующие алгоритмы для построения и  визуализации трехмерного ландшафта. 
%Согласно с оценками, представленными в таблицах \ref{tbl:dataLandscapePresent} -- \ref{tbl:drawAlgs}, были выбраны следующие алгоритмы: 
% 
%\begin{itemize}[label*=--]
%	\item в качестве способа представления данных о ландшафте была выбрана регулярная сетка высот;
%	\item в качестве алгоритма генерации карты высот был выбран алгоритм шума Перлина;
%
%\clearpage
%
%	\item в качестве алгоритма удаления невидимых линий и поверхностей был выбран алгоритм, использующий Z-буфер;
%	\item в качестве модели освещения и алгоритма закраски были выбраны модель Ламберта и алгоритм закраски по Гуро.
%\end{itemize}


                                                                                                                                                                                                     