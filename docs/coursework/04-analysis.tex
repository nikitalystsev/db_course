\chapter{Аналитический раздел}

\section{Анализ предметной области}

\subsection{Росстат}

\textbf{Росстат} ---  федеральный орган исполнительной власти, осуществляющий функции по формированию официальной статистической информации о социальных, экономических, демографических, экологических и других общественных процессах в Российской Федерации. \cite{info_rosstat}.

Одним из направлений сбора статистики являются \texttt{цены, инфляция}. Организации и их филиалы, индивидуальные предприниматели предоставляют в Росстат статистические отчеты, на основе которых производится расчет цен на различного рода товары и услуги.

\subsection{Элементы продуктовой корзины первой необходимости}

\textbf{Потребительская корзина (ПК)} --- это минимальный набор продуктов питания и непродовольственных товаров и услуг, необходимых для сохранения здоровья человека и обеспечения его жизнедеятельности\cite{info_consumer_basket2}. ПК помогает вести статистику в масштабах страны: сколько население должно потреблять и каковы реальные объёмы потребления.

ПК формируется отдельно для детей, трудоспособного населения и пенсионеров. Росстат собирает и обрабатывает информацию о фактических основных расходах в малоимущих семьях, а затем соотносит с рекомендациями учёных по минимальным потребностям человека.

Потребительская корзина состоит из 3-х частей:

\begin{itemize}[label=--]
	\item продукты питания;
	\item непродовольственные товары;
	\item услуги.
\end{itemize}

\clearpage

К продуктам питания следующие 11 групп товаров \cite{info_consumer_basket2}:

\begin{itemize}[label=--]
	\item хлебные продукты (хлеб и макаронные изделия в пересчете на муку, мука, крупы, бобовые);
	\item картофель;
	\item овощи и бахчевые;
	\item фрукты свежие;
	\item cахар и кондитерские изделия в пересчете на сахар;
	\item мясопродукты;
	\item рыбопродукты;
	\item молоко и молокопродукты в пересчете на молоко;
	\item яйца;
	\item масло растительное, маргарин и другие жиры;
	\item прочие продукты (соль, чай, специи).
\end{itemize}

\subsection{ФГИС <<Меркурий>>}

\textbf{ФГИС <<Меркурий>>} --- автоматизированная система, предназначенная для отслеживания продуктов питания на всей цепи производства и перемещения до точки реализации.

Работа с <<Меркурием>> заключается в создании и <<гашении>> \textbf{ветеринарно-сопроводительных документов (ВСД)} на всех этапах движения товара: от производства и переработки до продажи или утилизации.

\clearpage

Создание ФГИС <<Меркурий>> позволило достичь следующих целей \cite{info_mercury, info_mercury2}:

\begin{itemize}[label=--]
	\item защита потребителя от некачественной и небезопасной продукции, а все население страны от экономических и социальных угроз;
	\item обеспечение прозрачности и эффективности действий надзорных органов в борьбе с мошенничеством;
	\item минимизация бюрократии и предоставление удобного прозрачного механизма для комфортной работы частного бизнеса.
\end{itemize}

\section{Существующие решения}

На рынке существует большое количество сервисов для мониторинга цен на продукты в различных магазинах. Наиболее популярными являются:

\begin{itemize}[label=--]
	\item <<Едадил>>;
	\item SkidkaOnline;
	\item Price.ru.
\end{itemize}

Критерии, по которым будет произведено сравнение:

\begin{itemize}[label=--]
	\item \textbf{K1} --- возможность цены на конкретный товар в разных магазинах;
	\item \textbf{K2} --- возможность оставить отзыв о товаре;
	\item \textbf{K3} --- покрытие магазинов;
	\item \textbf{K4} --- наличие информации об акциях на товары;
	\item \textbf{K5} --- возможность просмотра динамики изменения цены.
\end{itemize}

\clearpage

\begin{table}[ht]
	\begin{center}
		\begin{threeparttable}
			\caption{Сравнение существующих решений}
			\label{tbl:exist_sol}
			\begin{tabular}{|c|c|c|c|c|c|}
				\hline
				\textbf{Решения} & \textbf{K1} & \textbf{K2} & \textbf{K3} & \textbf{K4} & \textbf{K5} \\
				\hline
				<<Едадил>> & + & + & Высокое & + & - \\
				\hline
				SkidkaOnline & - & + & Высокое & + & - \\
				\hline
				Price.ru & + & + & Низкое & + & + \\
				\hline
			\end{tabular}
		\end{threeparttable}			
	\end{center}
\end{table}

\section{Формализация и описание информации, подлежащей хранению в проектируемой базе данных}

Разрабатываемая база данных должна позволять хранить всю необходимую информацию для проведения сравнительного анализа цен на элементы продуктовой корзины первой необходимости. 

На основе анализа предметной области, в разрабатываемой базе данных были выделены сущности, приведенные в таблице \ref{tbl:db_entities}: 

\begin{table}[ht]
	\begin{center}
		\begin{threeparttable}
			\caption{Выделенные сущности предметной области и их описание}
			\label{tbl:db_entities}
			\begin{tabular}{|p{4.5cm}|p{10cm}|}
				\hline
				\textbf{Категория} & \textbf{Сведения} \\ \hline
				Магазин & Название, телефон, адрес, ФИО директора \\ 
				\hline
				Товар & Название, тип упаковки, вес, бренд, категория, масса нетто, состав \\ 
				\hline
				Сертификат соответствия & Тип сертификата, номер сертификата, нормативный документ, дата регистрации сертификата, дата окончания действия сертификата, статус соответствия \\ 
				\hline
				Цена & Цена, единица измерения, дата установки  \\ 
				\hline
				Акция & Тип акции, дата начала, дата конца, размер бонуса, описание \\ 
				\hline
				Оценка товара & Отзыв, рейтинг \\ 
				\hline
				Производитель & Название, страна \\ 
				\hline
				Дистрибьютор & Название, адрес, контактная информация \\ 
				\hline
				Ритейлер & Название, адрес, контактная информация \\ 
				\hline
			\end{tabular}
		\end{threeparttable}
	\end{center}
\end{table}

\clearpage

На рисунке \ref{img:er-diag} представлена ER-диаграмма сущностей проектируемой базы данных в нотации Чена:

\includesvgimage
{er-diag} % Имя файла без расширения (файл должен быть расположен в директории inc/img/)
{f} % Обтекание (без обтекания)
{h} % Положение рисунка (см. figure из пакета float)
{1\textwidth} % Ширина рисунка
{ER-диаграмма сущностей проектируемой базы данных в нотации Чена} % Подпись рисунка

\section{Формализация и описание пользователей проектируемого приложения к базе данных}

Разрабатываемое приложение должно предоставлять интерфейс для взаимодействия с базой данных. 

В соответствии выделенными в разрабатываемой базе данных сущностями выделяется 3 типа пользователей, описание которых приведено в таблице \ref{tbl:db_roles}:


\begin{table}[ht]
	\begin{center}
		\begin{threeparttable}
			\caption{Типы пользователей и их описание}
			\label{tbl:db_roles}
			\begin{tabular}{|p{4.5cm}|p{10cm}|}
				\hline
				\textbf{Тип пользователя} & \textbf{Описание} \\ \hline
				Обычный пользователь & Пользователь, имеющий меньше всего прав в системе. Может просматривать всю информацию о товарах, сравнивать цены, а также ставить оценки на товары. \\
				\hline
				<<Бабушка>> & Пользователь может изменять цены на товары в магазинах, добавлять новые магазины, добавлять новые товары. \\ 
				\hline
				Администратор & Пользователь с наивысшими правами доступа. Управляет всей системой. Может добавлять/удалять магазины, сертификаты на товары. \\ 
				\hline
			\end{tabular}
		\end{threeparttable}
	\end{center}
\end{table}


\section{Классификация и выбор СУБД по модели данных}

СУБД --- приложение, обеспечивающее создание, хранение, обновление и поиск информации в базах данных.

Модель данных --- это абстрактное и логическое определение объектов и его поведение, в совокупности составляющих доступ к данным, с которой взаимодействует пользователь~\cite{info_intro_db_williams}. С помощью модели данных могут быть представлены объекты предметной области и взаимосвязи между ними.

По модели данных СУБД разделяются на:

\begin{enumerate}[label={\arabic*)}]
	\item дореляционные модели, которые, в свою очередь, делятся на:
	\begin{itemize}[label*=--]
		\item инвертированные списки;
		\item иерархические;
		\item сетевые.
	\end{itemize}
	\item реляционные модели данных;
	\item постреляционные модели данных.
\end{enumerate}

\subsection{Дореляционные базы данных}

Дореляционные БД подразделяются на инвертированные списки (файлы), иерархические БД и сетевые БД.

БД на основе инвертированных списков --- это набор файлов с записями. Каждый файл имеет свой порядок, определяемый физической организацией данных. Для каждого файла могут быть созданы дополнительные упорядочения на основе значений полей записей (инвертированные списки), обычно с помощью индексов. В такой модели отсутствуют ограничения целостности, но их определяют программы, работающие с БД. Одно из немногих возможных ограничений --- уникальный индекс.

Иерархическая модель БД состоит из объектов с указателями от родительских объектов к потомкам, соединяя вместе связанную информацию. Иерархические БД могут быть представлены как дерево. 

К основным понятиям сетевой модели БД относятся: элемент (узел), связь. Узел --- это совокупность атрибутов данных, описывающих некоторый объект. Сетевые БД могут быть представлены в виде графа. В сетевой БД логика процедуры выборки данных зависит от физической организации этих данных. Поэтому эта модель не является полностью независимой от приложения. Другими словами, если необходимо изменить структуру данных, то нужно изменить и приложение. 

Преимущество дореляционных БД состоит в том, что они позволяют управлять данными на низком уровне. Недостатком является необходимость знать физическую организацию данных и зависимость прикладных систем от это организации \cite{info_db_kuznecov}.

\clearpage

\subsection{Реляционные базы данных}

Реляционная модель включает 3 компонента: 

\begin{enumerate}[label={\arabic*)}]
	\item структурный --- данные в базе данных представляют собой набор отношений;
	\item целостный --- отношения (таблицы) отвечают определенным условиям целостности;
	\item манипуляционный --- манипулирования отношениями осуществляется средствами реляционной алгебры и/или реляционного исчисления.
\end{enumerate}

Основными достоинствами реляционных БД является наличие небольшого набора абстракций, наличие простого и в то же время мощного математического аппарата, возможность ненавигационного манипулирования данными без необходимости знания конкретной физической организации баз данных во внешней памяти \cite{info_db_kuznecov}.

\subsection{Постреляционные базы данных}

Постреляционная модель данных в общем случае представляет собой расширенную реляционную модель, снимающую ограничение неделимости значений полей. То есть, допускаются многозначные поля, значения которых состоят из подзначений. Набор значений многозначных полей считается самостоятельной таблицей, встроенной в основную \cite{info_db_sopchenko}.

Достоинство постреляционной модели данных: возможность представления связанных реляционных таблиц одной постреляционной таблицей.  Недостаток постреляционной модели: сложность в обеспечении целостности данных \cite{info_db_sopchenko}.
              
\clearpage
                       
\section*{Вывод} 

В данном разделе была проанализирована предметная область, рассмотрены существующие решения. На основе анализа предметной области была формализована задача и данные, описаны типы пользователей. Для решения задачи была выбрана реляционная модель, так как для разрабатываемой базы данных важна простота хранения структурированных данных, целостность хранимых данных.