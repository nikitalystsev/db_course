\chapter{\label{implementation}Технологический раздел}

В данном разделе будет сделан обоснованный выбор средств реализации, будут представлены листинги команд для создания базы данных, таблиц базы данных, ограничений целостности. Также будут описаны способы тестирования триггеров и функций базы данных и будет описан интерфейс доступа к базе данных.

\section{Средства реализации}

В данном подразделе будут выбраны средства реализации базы данных и приложения.

\subsection{Выбор СУБД для реализации базы данных}

В качестве СУБД для реализации проектируемой базы данных рассматривались следующие варианты: 

\begin{itemize}
	\item \textbf{MySQL}. Данная СУБД имеет бесплатную версию с ограниченным функционалом и обладает простым синтаксисом, но не полностью соответствует стандартам SQL \cite{info_cmp_dbms};
	\item \textbf{Oracle} -- мультимодельная СУБД с подробной технической документацией. Бесплатные версии имеют очень ограниченный функционал, а платные имеют высокую стоимость \cite{info_cmp_dbms};
	\item \textbf{MS SQL} -- коммерческая СУБД с высокой стоимостью распространения. Имеет полную техническую документацию \cite{info_cmp_dbms};
	\item \textbf{PostgreSQL} -- СУБД с открытым исходным кодом и полной технической документацией \cite{info_cmp_dbms}. 
\end{itemize}

В таблице \ref{tbl:cmp_dbms} приведено сравнение рассматриваемых СУБД.

\begin{table}[H]
	\begin{center}
		\begin{threeparttable}
			\caption{Сравнение СУБД}
			\label{tbl:cmp_dbms}
			\begin{tabular}{|p{7cm}|c|c|c|c|}
				\hline
				\textbf{Критерии} & MySQL & Oracle & MS SQL & PostgreSQL \\
				\hline
				бесплатное распространение СУБД & + & - & - & + \\
				\hline
				наличие подробной документации & + & + & + & + \\
				\hline
				наличие опыта работы с СУБД & - & - & -  & + \\
				\hline
			\end{tabular}
		\end{threeparttable}			
	\end{center}
\end{table}

По результатам сравнения выбор был сделан в пользу СУБД PostgreSQL.

\subsection{Выбор средств реализации приложения}

В качестве типа приложения был сделан выбор в пользу Web приложения.

Для реализации backend и frontend частей приложения был выбран язык программирования Go \cite{info_go} по следующим причинам: 

\begin{itemize}
	\item Go обладает самым быстрым временем компиляции среди других языков, а также одним из лучших показателей производительности \cite{info_cmp_go, info_cmp_go_2};
	\item в стандартной библиотеке языка Go есть библиотека net\textbackslash http \cite{info_go_http_pkg}, предоставляющая весь необходимый функционал для создания http-сервера и маршрутизации и библиотека database\textbackslash sql \cite{info_go_sql_pkg}, предоставляющая весь необходимый функционал для работы с SQL базами данных;
	\item наличие опыта работы с данным языком в рамкам дисциплины <<Проектирование программного обеспечения>>.
\end{itemize}

В качестве среды разработки приложения была выбрана GoLand \cite{info_GoLand} от компании JetBrains \cite{info_JetBrains} по следующим причинам: 

\begin{itemize}
	\item данная среда разработки создана специально для написания приложений на Go \cite{info_GoLand_doc};
	\item в данной среде разработки <<из коробки>> предоставляются поддержки работы с различными типами СУБД, Git-ом \cite{info_Git} и Docker-ом \cite{info_Docker};
\end{itemize}

\clearpage 

\section{Реализация}

В данном подразделе будут описаны сущности реализованной базы данных, ограничения целостности, реализации триггера пересчета рейтинга продаваемого товара и функции получения среднего рейтинга, а также описание реализованной ролевой модели на уровне базы данных.

\subsection{Создание базы данных и сущностей базы данных}

На рисунках \ref{lst:create-db.txt} и \ref{lst:create-schema.txt} представлены скрипты для создания базы данных и схемы в базе данных.

\includelisting
{create-db.txt} % Имя файла с расширением (файл должен быть расположен в директории inc/lst/)
{Команда для создания базы данных} % Подпись листинга

\includelisting
{create-schema.txt} % Имя файла с расширением (файл должен быть расположен в директории inc/lst/)
{Команда для создания схемы в базе данных} % Подпись листинга

На рисунках \ref{lst:create-table-retailer.txt}-\ref{lst:create-table-distributor-manufacturer.txt} 
представлены скрипты для создания сущностей базы данных.

\includelisting
{create-table-retailer.txt} % Имя файла с расширением (файл должен быть расположен в директории inc/lst/)
{Команда создания таблицы Retailer} % Подпись листинга

\includelisting
{create-table-distributor.txt} % Имя файла с расширением (файл должен быть расположен в директории inc/lst/)
{Команда создания таблицы Distributor} % Подпись листинга

\includelisting
{create-table-manufacturer.txt} % Имя файла с расширением (файл должен быть расположен в директории inc/lst/)
{Команда создания таблицы Manufacturer} % Подпись листинга

\includelisting
{create-table-shop.txt} % Имя файла с расширением (файл должен быть расположен в директории inc/lst/)
{Команда создания таблицы Shop} % Подпись листинга

\includelisting
{create-table-product.txt} % Имя файла с расширением (файл должен быть расположен в директории inc/lst/)
{Команда создания таблицы Product} % Подпись листинга

\clearpage 

\includelisting
{create-table-certificate-compliance.txt} % Имя файла с расширением (файл должен быть расположен в директории inc/lst/)
{Команда создания таблицы Certificate\_compliance} % Подпись листинга

\includelisting
{create-table-user.txt} % Имя файла с расширением (файл должен быть расположен в директории inc/lst/)
{Команда создания таблицы User} % Подпись листинга

\includelisting
{create-table-promotion.txt} % Имя файла с расширением (файл должен быть расположен в директории inc/lst/)
{Команда создания таблицы Promotion} % Подпись листинга

\clearpage

\includelisting
{create-table-sale-product.txt} % Имя файла с расширением (файл должен быть расположен в директории inc/lst/)
{Команда создания таблицы SaleProduct} % Подпись листинга

\includelisting
{create-table-rating.txt} % Имя файла с расширением (файл должен быть расположен в директории inc/lst/)
{Команда создания таблицы Rating} % Подпись листинга

\includelisting
{create-table-retailer-distributor.txt} % Имя файла с расширением (файл должен быть расположен в директории inc/lst/)
{Команда создания таблицы RetailerDistributor} % Подпись листинга

\includelisting
{create-table-distributor-manufacturer.txt} % Имя файла с расширением (файл должен быть расположен в директории inc/lst/)
{Команда создания таблицы DistributorManufacturer} % Подпись листинга

\clearpage

\subsection{Создание ограничений целостности}

На листингах \ref{lst:create-table-constraints-retailer.txt}-\ref{lst:create-table-constraints-distributor-manufacturer.txt} представлены команды создания ограничений целостности всех таблиц базы данных.
\includelisting
{create-table-constraints-retailer.txt} % Имя файла с расширением (файл должен быть расположен в директории inc/lst/)
{Команды создания ограничений целостности для таблицы Retailer} % Подпись листинга

\includelisting
{create-table-constraints-distributor.txt} % Имя файла с расширением (файл должен быть расположен в директории inc/lst/)
{Команды создания ограничений целостности для таблицы Distributor} % Подпись листинга

\includelisting
{create-table-constraints-manufacturer-part1.txt} % Имя файла с расширением (файл должен быть расположен в директории inc/lst/)
{Команды создания ограничений целостности для таблицы Manufacturer (начало)} % Подпись листинга

\includelisting
{create-table-constraints-manufacturer-part2.txt} % Имя файла с расширением (файл должен быть расположен в директории inc/lst/)
{Команды создания ограничений целостности для таблицы Manufacturer (конец)} % Подпись листинга

\includelisting
{create-table-constraints-shop.txt} % Имя файла с расширением (файл должен быть расположен в директории inc/lst/)
{Команды создания ограничений целостности для таблицы Shop} % Подпись листинга

\includelisting
{create-table-constraints-product-part1.txt} % Имя файла с расширением (файл должен быть расположен в директории inc/lst/)
{Команды создания ограничений целостности для таблицы Product (начало)} % Подпись листинга

\includelisting
{create-table-constraints-product-part2.txt} % Имя файла с расширением (файл должен быть расположен в директории inc/lst/)
{Команды создания ограничений целостности для таблицы Product (конец)} % Подпись листинга

\includelisting
{create-table-constraints-certificate-compliance.txt} % Имя файла с расширением (файл должен быть расположен в директории inc/lst/)
{Команды создания ограничений целостности для таблицы Certificate\_compliance} % Подпись листинга

\includelisting
{create-table-constraints-user.txt} % Имя файла с расширением (файл должен быть расположен в директории inc/lst/)
{Команды создания ограничений целостности для таблицы User} % Подпись листинга

\includelisting
{create-table-constraints-promotion.txt} % Имя файла с расширением (файл должен быть расположен в директории inc/lst/)
{Команды создания ограничений целостности для таблицы Promotion} % Подпись листинга

\clearpage

\includelisting
{create-table-constraints-sale-product.txt} % Имя файла с расширением (файл должен быть расположен в директории inc/lst/)
{Команды создания ограничений целостности для таблицы SaleProduct} % Подпись листинга

\includelisting
{create-table-constraints-rating.txt} % Имя файла с расширением (файл должен быть расположен в директории inc/lst/)
{Команды создания ограничений целостности для таблицы Rating} % Подпись листинга

\clearpage

\includelisting
{create-table-constraints-retailer-distributor.txt} % Имя файла с расширением (файл должен быть расположен в директории inc/lst/)
{Команды создания ограничений целостности для таблицы RetailerDistributor} % Подпись листинга

\includelisting
{create-table-constraints-distributor-manufacturer.txt} % Имя файла с расширением (файл должен быть расположен в директории inc/lst/)
{Команды создания ограничений целостности для таблицы DistributorManufacturer} % Подпись листинга

\clearpage

\subsection{Создание триггеров и функций базы данных}

На рисунке \ref{img:trigger-avg-rating-impl.pdf} изображена реализация триггера для обновления рейтинга продаваемого товара в формате схемы алоритма.

\includepdfimage
{trigger-avg-rating-impl.pdf} % Имя файла без расширения (файл должен быть расположен в директории inc/img/)
{f} % Обтекание (без обтекания)
{h} % Положение рисунка (см. figure из пакета float)
{1\textwidth} % Ширина рисунка
{Реализация триггера для обновления среднего рейтинга} % Подпись рисунка

\clearpage

На рисунке \ref{img:func-get-avg-rating-impl.pdf} изображена реализация функции для получения среднего рейтинга продаваемого товара в формате схемы алоритма.

\includepdfimage
{func-get-avg-rating-impl.pdf} % Имя файла без расширения (файл должен быть расположен в директории inc/img/)
{f} % Обтекание (без обтекания)
{h} % Положение рисунка (см. figure из пакета float)
{1\textwidth} % Ширина рисунка
{Реализация функции для получения среднего рейтинга} % Подпись рисунка

На листинге \ref{lst:create-trigger-avg-rating.txt} показаны команды для создания триггера на обновление рейтинга продаваемого товара.

\includelisting
{create-trigger-avg-rating.txt} % Имя файла с расширением (файл должен быть расположен в директории inc/lst/)
{Команды создания триггера на обновления среднего рейтинга продаваемого товара} % Подпись листинга

\clearpage

\subsection{Создание ролевой модели}

На листингах \ref{lst:create-role-guest.txt}-\ref{lst:create-role-admin.txt} представлено создание ролевой модели на уровне базы данных.

\includelisting
{create-role-guest.txt} % Имя файла с расширением (файл должен быть расположен в директории inc/lst/)
{Команды создания роли гостя} % Подпись листинга

\includelisting
{create-role-registered.txt} % Имя файла с расширением (файл должен быть расположен в директории inc/lst/)
{Команды создания роли зарегистрированного пользователя} % Подпись листинга

\includelisting
{create-role-admin.txt} % Имя файла с расширением (файл должен быть расположен в директории inc/lst/)
{Команды создания роли администратора} % Подпись листинга

На листинге \ref{lst:create-users.txt} представлено создание пользователей базы данных.

\includelisting
{create-users.txt} % Имя файла с расширением (файл должен быть расположен в директории inc/lst/)
{Команды создания пользователей} % Подпись листинга

\clearpage

\section{Тестирование созданных триггеров и функций базы данных}

В данном подразделе будут описаны тесты для функции получения среднего рейтинга продаваемого товара и тесты для триггера обновления среднего рейтинга при добавлении или удалении 

Для триггеров и функций были написаны unit тесты \cite{info_unit_tests} c использованием расширения pgTAP \cite{info_pgTAB} для PostgreSQL.

\subsection{Тестирование функции}

Для тестирования функции получения срелнего рейтинга продаваемого товара было написано 5 unit тестов -- 2 позитивных и 3 негативных:

\begin{enumerate}
	\item у продаваемого товара нет отзывов. Ожидаемое значение среднего рейтинга -- null;
	\item у продаваемого товара есть 2 отзыва. Ожидаемое значение среднего рейтинга -- 3.5;
	\item передача пустой строки в качестве id записи в таблице sale\_product. Ожидаемое значение -- исключение;
	\item не переданы аргументы в тестируемую функцию. Ожидаемое значение -- исключение;
	\item передан некорректный аргумент в тестируемую функцию. Ожидаемое значение -- исключение.
\end{enumerate}

\includelisting
{func-unit-tests-part1.txt} % Имя файла с расширением (файл должен быть расположен в директории inc/lst/)
{Тестовые случаи для функции получения среднего рейтинга продаваемого товара (начало)} % Подпись листинга

\includelisting
{func-unit-tests-part2.txt} % Имя файла с расширением (файл должен быть расположен в директории inc/lst/)
{Тестовые случаи для функции получения среднего рейтинга продаваемого товара (конец)} % Подпись листинга

Все тесты были пройдены успешно.

\clearpage

\subsection{Тестирование триггера}

Для тестирования триггера обновления среднего рейтинга продаваемого товара было написано 2 unit теста:

\begin{enumerate}
	\item у продаваемого товара нет отзывов. Добавляется один отзыв с рейтингом, равным 5. Ожидаемое значение среднего рейтинга -- 5;
	\item у продаваемого товара есть один отзыв. Отзыв удаляется. Ожидаемое значение среднего рейтинга -- null;
\end{enumerate}

\includelisting
{trigger-unit-tests.txt} % Имя файла с расширением (файл должен быть расположен в директории inc/lst/)
{Тестовые случаи для триггера обновления рейтинга продаваемого товара} % Подпись листинга

Все тесты были пройдены успешно

\section{Интерфейс доступа к базе данных}

Для взаимодействия пользователей с базой данных было написано монолитное Web-приложение c использованием архитектурного подхода REST API \cite{info_rest_api} и консольным интерфейсом.

На рисунках \ref{img:main-menu}-\ref{img:shop-menu} представлены интерфейсы приложения для взаимодействия с базой данных.

\includeimage
{main-menu} % Имя файла без расширения (файл должен быть расположен в директории inc/img/)
{f} % Обтекание (без обтекания)
{h} % Положение рисунка (см. figure из пакета float)
{1\textwidth} % Ширина рисунка
{Интерфейс главного меню приложения} % Подпись рисунка

Интерфейс главного меню позволяет зарегистрировать нового пользователя в системе, войти в систему как пользователь и как администратор, а также перейти к каталогу с товарами и остановить работу интерфейса.

\includeimage
{product-catalog-menu} % Имя файла без расширения (файл должен быть расположен в директории inc/img/)
{f} % Обтекание (без обтекания)
{h} % Положение рисунка (см. figure из пакета float)
{1\textwidth} % Ширина рисунка
{Интерфейс меню каталога} % Подпись рисунка

Интерфейс меню каталога позволяет вывести страницу товаров (10 элементов на странице) и перейти к выбранному товару из списка выведенных страниц.

\clearpage

\includeimage
{product-menu} % Имя файла без расширения (файл должен быть расположен в директории inc/img/)
{f} % Обтекание (без обтекания)
{h} % Положение рисунка (см. figure из пакета float)
{1\textwidth} % Ширина рисунка
{Интерфейс меню для обработки действий с товаром} % Подпись рисунка

Интерфейс меню для обработки действий с товаром позволяет сравнить цены на товар в  различных магазинах, посмотреть, каким сертификатам соответствует товар.

\includeimage
{admin-product-menu} % Имя файла без расширения (файл должен быть расположен в директории inc/img/)
{f} % Обтекание (без обтекания)
{h} % Положение рисунка (см. figure из пакета float)
{1\textwidth} % Ширина рисунка
{Интерфейс меню для обработки действий с товаром для администратора} % Подпись рисунка

Администратор получает расширенный набор опций для взаимодействия с товаром: к уже существующим добавляются возможности добавления нового сертификата на товар, удаления уже имеющегося сертификата и обновления статуса соответствия сертификата.

\includeimage
{user-main-menu} % Имя файла без расширения (файл должен быть расположен в директории inc/img/)
{f} % Обтекание (без обтекания)
{h} % Положение рисунка (см. figure из пакета float)
{1\textwidth} % Ширина рисунка
{Интерфейс главного меню аутентифицированного пользователя} % Подпись рисунка

\clearpage 

Интерфейс главного меню аутентифицированного пользователя позволяет перейти в каталог товаров, добавить новый магазин, а также перейти к обработке магазинов.

\includeimage
{shop-catalog-menu} % Имя файла без расширения (файл должен быть расположен в директории inc/img/)
{f} % Обтекание (без обтекания)
{h} % Положение рисунка (см. figure из пакета float)
{1\textwidth} % Ширина рисунка
{Интерфейс меню для обработки магазинов} % Подпись рисунка

Аутентифицированный пользователь в интерфейсе меню для обработки магазинов может вывести страницу магазинов и перейти к конкретному магазину.

\includeimage
{admin-shop-catalog-menu} % Имя файла без расширения (файл должен быть расположен в директории inc/img/)
{f} % Обтекание (без обтекания)
{h} % Положение рисунка (см. figure из пакета float)
{1\textwidth} % Ширина рисунка
{Интерфейс меню для обработки магазинов для администратора} % Подпись рисунка

Администратор, помимо выше изложенных возможностей, также может удалить конкретный магазин.

\clearpage

\includeimage
{shop-menu} % Имя файла без расширения (файл должен быть расположен в директории inc/img/)
{f} % Обтекание (без обтекания)
{h} % Положение рисунка (см. figure из пакета float)
{1\textwidth} % Ширина рисунка
{Интерфейс меню для обработки магазина} % Подпись рисунка

Интерфейс меню для обработки магазина позволяет аутентифицированному пользователю добавить новый товар,  оценить товар и изменить его цену.


\section*{Вывод}

В данном разделе был сделан обоснованный выбор средств реализации, были представлены листинги команд для создания базы данных, таблиц базы данных, ограничений целостности. Также были описаны способы тестирования триггеров и функций базы данных и был описан интерфейс доступа к базе данных.
