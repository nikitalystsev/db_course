\chapter{\label{design}Конструкторский раздел}

В данном разделе будет представлена диаграмма проектируемой базы данных, будут описаны ее сущности, а также будут описаны требуемые ограничения целостности. Будут представлены схемы алгоритмов триггеров и функций.

\section{Таблицы базы данных}

На рисунке \ref{img:er-entities-diag.drawio.pdf} представлена ER-модель разрабатывамой базы данных:

\includepdfimage
{er-entities-diag.drawio.pdf} % Имя файла без расширения (файл должен быть расположен в директории inc/img/)
{f} % Обтекание (без обтекания)
{h} % Положение рисунка (см. figure из пакета float)
{0.9\textwidth} % Ширина рисунка
{ER-модель базы данных} % Подпись рисунка

\clearpage 

\section{Описание сущностей}

В соответствии с диаграммой проектируемой БД были составлены следующие сущности:

\begin{enumerate}
	\item \textbf{User} -- представляет пользователя системы. Содержит следующие поля: 
	\begin{itemize}
		\item id -- уникальный идентификатор пользователя;
		\item fio -- ФИО пользователя;
		\item phone\_number -- номер телефона пользователя;
		\item password -- поле, хранящее хешированный пароль аккаунта пользователя.
		\item registration\_date -- дата  регистрации пользователя.
	\end{itemize}
	
	\item \textbf{Shop} -- представляет объект магазина. Содержит следующие поля: 
	\begin{itemize}
		\item id -- уникальный идентификатор магазина;
		\item retailer\_id -- идентификатор ритейлера, с которым работает сотрудничает магазин;
		\item title -- название магазина;
		\item address -- адрес магазина;
		\item phone\_number -- номер телефона магазина;
		\item fio\_director -- ФИО директора магазина;
	\end{itemize}
	
	\item \textbf{Product} -- представляет объект товара. Содержит следующие поля: 
	\begin{itemize}
		\item id -- уникальный идентификатор магазина;
		\item retailer\_id -- идентификатор ритейлера, который реализует товар;
		\item distributor\_id -- идентификатор дистрибьютора, который распространяет товар;
		\item manufacturer\_id -- идентификатор производителя, который производит товар;
		\item name -- название товара;
		\item categories -- категория товара;
		\item brand -- бренд товара;
		\item compound -- состав товара;
		\item gross\_mass -- масса брутто;
		\item net\_mass -- масса нетто;
		\item package\_type -- тип упаковки.
	\end{itemize}
	
	\item \textbf{Certificate\_compliance} -- представляет объект сертификата соответствия. Содержит следующие поля: 
	\begin{itemize}
		\item id -- уникальный идентификатор сертификата;
		\item product\_id -- идентификатор товара, который соответствует данному сертификату;
		\item type -- тип сертификата;
		\item number -- номер сертификата;
		\item normative\_document -- нормативный документ, которому удовлетворяет товар;
		\item status\_compliance -- статус соответствия;
		\item registration\_data -- дата регистрации сертификата;
		\item expiration\_date -- дата окончания действия сертификата.
	\end{itemize}
	
	\item \textbf{Promotion} -- представляет объект акции на товар. Содержит следующие поля: 
	\begin{itemize}
		\item id -- уникальный идентификатор акции;
		\item type -- тип акции;
		\item description -- описание акции;
		\item discount\_size -- размер скидки в процентах;
		\item start\_date -- дата начала акции;
		\item end\_date -- дата конца акции.
	\end{itemize}
	
\clearpage 

	\item \textbf{Rating} -- представляет объект оценки товара. Содержит следующие поля: 
	\begin{itemize}
		\item id -- уникальный идентификатор оценки;
		\item user\_id -- идентификатор пользователя, оставившего отзыв;
		\item  sale\_product\_id -- идентификатор продажи конкретного товара в конкретном магазине;
		\item review -- отзыв;
		\item rating -- рейтинг.
	\end{itemize}
	
	\item \textbf{Manufacturer} -- представляет объект производителя товара. Содержит следующие поля: 
	\begin{itemize}
		\item id -- уникальный идентификатор производителя;
		\item title -- название производителя;
		\item address -- адрес производителя;
		\item phone\_number -- номер телефона;
		\item fio\_representative -- ФИО представителя.
	\end{itemize}
	
	\item \textbf{Distributor} -- представляет объект дистрибьютора товара. Содержит следующие поля: 
	\begin{itemize}
		\item id -- уникальный идентификатор дистрибьютора;
		\item title -- название дистрибьютора;
		\item address -- адрес дистрибьютора;
		\item phone\_number -- номер телефона;
		\item fio\_representative -- ФИО представителя.
	\end{itemize}
	
	\item \textbf{Retailer} -- представляет объект ритейлера товара. Содержит следующие поля: 
	\begin{itemize}
		\item id -- уникальный идентификатор ритейлера;
		\item title -- название ритейлера;
		\item address -- адрес ритейлера;
		\item phone\_number -- номер телефона;
		\item fio\_representative -- ФИО представителя.
	\end{itemize}
	
	\item \textbf{SaleProduct} -- представляет объект продажи товара. Содержит следующие поля: 
	\begin{itemize}
		\item id -- уникальный идентификатор продажи;
		\item shop\_id -- идентификатор магазина, продающего товар;
		\item product\_id -- идентификатор продаваемого товара;
		\item promotion\_id -- идентификатор применяемой к товару акции (если она есть);
		\item price -- цена товара;
		\item currency -- валюта;
		\item setting\_date -- дата установки цены.
		\item avg\_rating -- средний рейтинг товара в магазине.
	\end{itemize}
	
	\item \textbf{ReatailerDistributor} -- реализует связь <<многие-ко-многим>> для отношений Reatailer и Distributor. Содержит следующие поля: 
	\begin{itemize}
		\item retailer\_id -- идентификатор ритейлера;
		\item distributor\_id -- идентификатор дистрибьютора.
	\end{itemize}
	
	\item \textbf{DistributorManufacturer} -- реализует связь <<многие-ко-многим>> для отношений Distributor и Manufacturer. Содержит следующие поля: 
	\begin{itemize}
		\item distributor\_id -- идентификатор дистрибьютора.
		\item manufacturer\_id -- идентификатор производителя.
	\end{itemize}
\end{enumerate}

\section{Ограничения целостности}

В таблицах \ref{tbl:entity_user} -- \ref{tbl:entity_distributor_manufacturer} приведены ограничения целостности для каждой таблицы разрабатываемой БД.
 
\clearpage 

\begin{table}[!h]
	\begin{center}
		\begin{threeparttable}
			\caption{Ограничения целостности таблицы User}
			\label{tbl:entity_user}
			\begin{tabular}{|p{4.5cm}|p{2.5cm}|p{8.5cm}|}
				\hline 
				\textbf{Поле} & \textbf{Тип} & \textbf{Ограничение}  \\
				\hline
				id & UUID & Не должно быть пустым. Является первичным ключом записи  \\
				\hline
				fio & строка & Не должно быть пустым  \\
				\hline
				phone\_number & строка & Не должно быть пустым, должно быть уникальным  \\
				\hline
				password & строка & Не должно быть пустым  \\
				\hline
				registration\_date & дата & Не должно быть пустым  \\
				\hline
			\end{tabular}
		\end{threeparttable}			
	\end{center}
\end{table}

\begin{table}[!h]
	\begin{center}
		\begin{threeparttable}
			\caption{Ограничения целостности таблицы Shop}
			\label{tbl:entity_shop}
			\begin{tabular}{|p{4.5cm}|p{2.5cm}|p{8.5cm}|}
				\hline 
				\textbf{Поле} & \textbf{Тип} & \textbf{Ограничение}  \\
				\hline
				id & UUID & Не должно быть пустым. Является первичным ключом записи  \\
				\hline
				retailer\_id & UUID & Не должно быть пустым. Внешний ключ на запись в таблице Retailer   \\
				\hline
				title & строка & Не должно быть пустым \\
				\hline
				address & строка & Не должно быть пустым, должно быть уникальным  \\
				\hline
				phone\_number & строка & Не должно быть пустым  \\
				\hline
				fio\_director & строка & Не должно быть пустым  \\
				\hline
			\end{tabular}
		\end{threeparttable}			
	\end{center}
\end{table}


\begin{table}[!h]
	\begin{center}
		\begin{threeparttable}
			\caption{Ограничения целостности таблицы Rating}
			\label{tbl:entity_rating}
			\begin{tabular}{|p{4.5cm}|p{2.5cm}|p{8.5cm}|}
				\hline 
				\textbf{Поле} & \textbf{Тип} & \textbf{Ограничение}  \\
				\hline
				id & UUID & Не должно быть пустым. Является первичным ключом записи  \\
				\hline
				user\_id & UUID & Не должно быть пустым. Внешний ключ на запись в User  \\
				\hline
				sale\_product\_id & UUID & Не должно быть пустым. Внешний ключ на запись в таблицу SaleProduct  \\
				\hline
				review & строка & Не должно быть пустым \\
				\hline
				rating & число & Не должно быть пустым. Не может быть меньше 0 и больше 5 \\
				\hline
			\end{tabular}
		\end{threeparttable}			
	\end{center}
\end{table}

\clearpage

\begin{table}[!h]
	\begin{center}
		\begin{threeparttable}
			\caption{Ограничения целостности таблицы Promotion}
			\label{tbl:entity_promotion}
			\begin{tabular}{|p{4.5cm}|p{2.5cm}|p{8.5cm}|}
				\hline 
				\textbf{Поле} & \textbf{Тип} & \textbf{Ограничение}  \\
				\hline
				id & UUID & Не должно быть пустым. Является первичным ключом записи  \\
				\hline
				type & строка & Не должно быть пустым  \\
				\hline
				description & строка & Не должно быть пустым  \\
				\hline
				discount\_size & чисто & Не должно быть больше не должно быть меньше 0 и больше 100 \\
				\hline
				start\_date & дата & Не должно быть пустым \\
				\hline
				end\_date & дата & Не должно быть пустым \\
				\hline
			\end{tabular}
		\end{threeparttable}			
	\end{center}
\end{table}

В таблице Promotion для полей start\_date и end\_date также определено следующее ограничение: $ start\_date < end\_date $.

\begin{table}[!h]
	\begin{center}
		\begin{threeparttable}
			\caption{Ограничения целостности таблицы Certificate\_compliance}
			\label{tbl:entity_сertificate_compliance}
			\begin{tabular}{|p{4.5cm}|p{2.5cm}|p{8.5cm}|}
				\hline 
				\textbf{Поле} & \textbf{Тип} & \textbf{Ограничение}  \\
				\hline
				id & UUID & Не должно быть пустым. Является первичным ключом записи  \\
				\hline
				product\_id & UUID & Не должно быть пустым. Внешний ключ на запись в таблице Product  \\
				\hline
				type & строка & Не должно быть пустым  \\
				\hline
				number & строка & Не должно быть пустым  \\
				\hline
				normative\_document & строка & Не должно быть пустым  \\
				\hline
				status\_compliance & boolean & Не должно быть пустым  \\
				\hline
				registration\_data & дата & Не должно быть пустым  \\
				\hline
				expiration\_date & дата & Не должно быть пустым  \\
				\hline
			\end{tabular}
		\end{threeparttable}			
	\end{center}
\end{table}

В таблице Certificate\_compliance для полей registration\_data и expiration\_date также определено следующее ограничение: $ registration\_data < expiration\_date $.

\clearpage

Для таблиц Retailer, Distributor, Manufacturer определены следующие ограничения полей: 

\begin{table}[!h]
	\begin{center}
		\begin{threeparttable}
			\caption{Ограничения целостности таблиц Retailer, Distributor и Manufacturer}
			\label{tbl:entity_rdm}
			\begin{tabular}{|p{4.5cm}|p{2.5cm}|p{8.5cm}|}
				\hline 
				\textbf{Поле} & \textbf{Тип} & \textbf{Ограничение}  \\
				\hline
				id & UUID & Не должно быть пустым. Является первичным ключом записи  \\
				\hline
				title & строка & Не должно быть пустым   \\
				\hline
				address & строка & Не должно быть пустым. Должно быть уникальным.  \\
				\hline
				phone\_number & строка & Не должно быть пустым. Должно быть уникальным. \\
				\hline
				fio\_representative & строка & Не должно быть пустым. Должно быть уникальным \\
				\hline
			\end{tabular}
		\end{threeparttable}			
	\end{center}
\end{table}

\begin{table}[!h]
	\begin{center}
		\begin{threeparttable}
			\caption{Ограничения целостности таблицы Product}
			\label{tbl:entity_product}
			\begin{tabular}{|p{4.5cm}|p{2.5cm}|p{8.5cm}|}
				\hline 
				\textbf{Поле} & \textbf{Тип} & \textbf{Ограничение}  \\
				\hline
				id & UUID & Не должно быть пустым. Является первичным ключом записи  \\
				\hline
				retailer\_id & UUID & Не должно быть пустым. Внешний ключ на запись в таблице Retailer  \\
				\hline
				distributor\_id & UUID & Не должно быть пустым. Внешний ключ на запись в таблице Distributor \\
				\hline
				manufacturer\_id & UUID & Не должно быть пустым. Внешний ключ на запись в таблице Manufacturer  \\
				\hline
				name & строка & Не должно быть пустым  \\
				\hline
				categories & строка & Не должно быть пустым  \\
				\hline
				brand & строка & Не должно быть пустым  \\
				\hline
				compound & строка & Не должно быть пустым  \\
				\hline
				gross\_mass & число & Не должно быть пустым. Не должно быть меньше 0  \\
				\hline
				net\_mass & число & Не должно быть пустым. Не должно быть меньше 0  \\
				\hline
				package\_type & строка & Не должно быть пустым  \\
				\hline
			\end{tabular}
		\end{threeparttable}			
	\end{center}
\end{table}

В таблице Product для полей gross\_mass и net\_mass также определено следующее ограничение: $ gross\_mass \geq net\_mass $.

\begin{table}[!h]
	\begin{center}
		\begin{threeparttable}
			\caption{Ограничения целостности таблицы SaleProduct}
			\label{tbl:entity_sale_product}
			\begin{tabular}{|p{4.5cm}|p{2.5cm}|p{8.5cm}|}
				\hline 
				\textbf{Поле} & \textbf{Тип} & \textbf{Ограничение}  \\
				\hline
				id & UUID & Не должно быть пустым. Является первичным ключом записи  \\
				\hline
				shop\_id & UUID & Не должно быть пустым. Внешний ключ на запись в таблице Shop  \\
				\hline
				product\_id & UUID & Не должно быть пустым. Внешний ключ на запись в таблице Product \\
				\hline
				promotion\_id & UUID & Не должно быть пустым. Внешний ключ на запись в таблице Promotion \\
				\hline
				price & число & Не должно быть пустым. Не должно быть меньше 0  \\
				\hline
				currency & строка & Не должно быть пустым  \\
				\hline
				setting\_date & дата & Не должно быть пустым  \\
				\hline
				avg\_rating & число & Не должно быть пустым. Не может быть меньше 0 и больше 5 \\
				\hline
			\end{tabular}
		\end{threeparttable}			
	\end{center}
\end{table}

\begin{table}[!h]
	\begin{center}
		\begin{threeparttable}
			\caption{Ограничения целостности таблицы RetailerDistributor}
			\label{tbl:entity_retailer_distributor}
			\begin{tabular}{|p{4.5cm}|p{2.5cm}|p{8.5cm}|}
				\hline 
				\textbf{Поле} & \textbf{Тип} & \textbf{Ограничение}  \\
				\hline
				retailer\_id & UUID & Не должно быть пустым. Внешний ключ на запись в таблице Retailer  \\
				\hline
				distributor\_id & UUID & Не должно быть пустым. Внешний ключ на запись в таблице Distributor \\
				\hline
			\end{tabular}
		\end{threeparttable}			
	\end{center}
\end{table}

В таблице RetailerDistributor также определено следующее ограничение: в совокупности поля retailer\_id и distributor\_id определяют первичный ключ записи.


\begin{table}[!h]
	\begin{center}
		\begin{threeparttable}
			\caption{Ограничения целостности таблицы DistributorManufacturer}
			\label{tbl:entity_distributor_manufacturer}
			\begin{tabular}{|p{4.5cm}|p{2.5cm}|p{8.5cm}|}
				\hline 
				\textbf{Поле} & \textbf{Тип} & \textbf{Ограничение}  \\
				\hline
				distributor\_id & UUID & Не должно быть пустым. Внешний ключ на запись в таблице Distributor \\
				\hline
				manufacturer\_id & UUID & Не должно быть пустым. Внешний ключ на запись в таблице Manufacturer  \\
				\hline
			\end{tabular}
		\end{threeparttable}			
	\end{center}
\end{table}

\clearpage 

В таблице DistributorManufacturer также определено следующее ограничение: в совокупности поля distributor\_id и manufacturer\_id определяют первичный ключ записи.

\section{Триггеры базы данных}

В разрабатываемой системе у каждого продающегося товара есть средний рейтинг, который определяется как среднее арифметическое рейтингов в отзывах пользователей. Для корректного отображения среднего рейтинга необходимо вручную делать запросы к базе данных Rating, считать средний рейтинг как сумму рейтингов, деленную на их количество, а затем обновлять поле rating в записи в таблице SaleProduct. Чтобы автоматизировать процесс пересчета рейтинга, можно написать after триггер, который будет выполнять описанные выше действия после добавления или удаления отзыва на товар. 

\clearpage

Схема работы данного триггера представлена на рисунке \ref{img:trigger-avg-rating.pdf}: 

\includepdfimage
{trigger-avg-rating.pdf} % Имя файла без расширения (файл должен быть расположен в директории inc/img/)
{f} % Обтекание (без обтекания)
{h} % Положение рисунка (см. figure из пакета float)
{1\textwidth} % Ширина рисунка
{Схема работы триггера для обновления среднего рейтинга продаваемого товара} % Подпись рисунка

\clearpage

\section{Функции базы данных}

На рисунке \ref{img:func-get-avg-rating.pdf} изображена схема алгоритма  функции для получения среднего рейтинга продаваемого товара.

\includepdfimage
{func-get-avg-rating.pdf} % Имя файла без расширения (файл должен быть расположен в директории inc/img/)
{f} % Обтекание (без обтекания)
{h} % Положение рисунка (см. figure из пакета float)
{1\textwidth} % Ширина рисунка
{Схема работы функции для получения среднего рейтинга продаваемого товара} % Подпись рисунка

\clearpage

\section{Описание проектируемой ролевой модели на уровне базы данных}

Разрабатываемая база данных должна содержать следующие роли со следующими правами доступа к таблицам:

\begin{enumerate}
	\item \textbf{гость}. Данной роли предоставляются права на чтение всех таблиц. Также пользователи данной роли имеют права на создание записей в таблице User;
	\item \textbf{зарегистрированный пользователь}. Данная роль имеет все права роли гость, а также права на создание записей в таблицах Promotion, SaleProduct, Rating, Shop, Product, Retailer, Distributor, Manufacturer, RetailerDistributor, DistributorManufacturer и права на обновление записей в таблице Price;
	\item \textbf{администратор}. Данная роль имеет все права в отношении всех таблиц базы данных.
\end{enumerate}

% добавить

%В данном разделе будет приведены требования к разрабатываемому ПО, будет приведено формальное описание алгоритмов построения и визуализации ландшафта, будет приведена структура разрабатываемого ПО.
%
%\section{Требования к программному обеспечению}
%
%Разрабатываемое программное обеспечение должно предоставлять пользователю следующую функциональность:
%
%\begin{itemize}[label=--]
%	\item изменение параметров алгоритма шума Перлина для генерации ландшафта;
%	\item изменение параметров и положения источника света;
%	\item возможность управления положением модели (перемещение, масштабирование, поворот);
%	\item задание и изменение уровня моря в интерактивном режиме;
%\end{itemize}
%
%При этом разрабатываемая программа должна удовлетворять следующим требованиям:
%
%\begin{itemize}[label=--]
%	\item время отклика программы должно быть менее 1 секунды для корректной работы в интерактивном режиме;
%	\item программа должна корректно реагировать на любые действия пользователя.
%\end{itemize}
%
%\section{Разработка алгоритмов}
%
%\subsection{Общий алгоритм построения изображения}
%
%На рисунке \ref{img:generalImageConstrAlg} представлена схема алгоритма построения одного кадра изображения.
%
%\includesvgimage
%{generalImageConstrAlg} % Имя файла без расширения (файл должен быть расположен в директории inc/img/)
%{f} % Обтекание (без обтекания)
%{h} % Положение рисунка (см. figure из пакета float)
%{1\textwidth} % Ширина рисунка
%{Схема алгоритма построения одного кадра изображения} % Подпись рисунка
%
%\clearpage
%
%\subsection{Алгоритм процедурной генерации ландшафта на основе шума Перлина}
%
%На рисунке \ref{img:genHeightMapAlg} представлена схема алгоритма генерации карты высот на основе шума Перлина.
%
%\includesvgimage
%{genHeightMapAlg} % Имя файла без расширения (файл должен быть расположен в директории inc/img/)
%{f} % Обтекание (без обтекания)
%{h} % Положение рисунка (см. figure из пакета float)
%{1\textwidth} % Ширина рисунка
%{Схема алгоритма генерации карты высот на основе шума Перлина} % Подпись рисунка
%
%\clearpage
%
%На рисунке \ref{img:perlinNoiseWithMultiOctavesAlg} представлена схема алгоритма шума Перлина с применением нескольких октав.
%
%\includesvgimage
%{perlinNoiseWithMultiOctavesAlg} % Имя файла без расширения (файл должен быть расположен в директории inc/img/)
%{f} % Обтекание (без обтекания)
%{h} % Положение рисунка (см. figure из пакета float)
%{0.65\textwidth} % Ширина рисунка
%{Схема алгоритма шума Перлина с применением нескольких октав} % Подпись рисунка
%
%\clearpage
%
%На рисунке \ref{img:perlinNoiseAlg} представлена схема алгоритма шума Перлина.
%
%\includesvgimage
%{perlinNoiseAlg} % Имя файла без расширения (файл должен быть расположен в директории inc/img/)
%{f} % Обтекание (без обтекания)
%{h} % Положение рисунка (см. figure из пакета float)
%{1\textwidth} % Ширина рисунка
%{Схема алгоритма шума Перлина} % Подпись рисунка
%
%\clearpage
%
%\subsection{Афинные преобразования}
%
%Для осуществления управлением положения модели используются афинные преобразования \cite{info_affineTransform}, задающиеся матрицами.
%Изменение положения точки трехмерного пространства в результате применение одного из афинных преобразований задается формулой \ref{mtr:affineTransform}.
%
%\begin{equation}
%	\label{mtr:affineTransform}
%	\begin{gathered}
%		(x', y', z', 1) = (x, y, z, 1) \cdot M
%	\end{gathered}
%\end{equation}
%
%где $x$, $y$, $z$ -- старые координаты точки, $x'$, $y'$, $z'$ -- новые координаты точки, $M$ -- матрица афинного преобразования (формулы \ref{mtr:OX} - \ref{mtr:scale}).
%
%Поворот вокруг одной из осей координат задается углом вращения $\alpha$ и осуществляется с помощью соответствующей матрицы поворота (формулы \ref{mtr:OX} - \ref{mtr:OZ}).
%
%Матрица поворота вокруг оси OX:
%
%\begin{equation}
%	\label{mtr:OX}
%	\begin{pmatrix}
%		1 	& 0 		  & 0 	       & 0 \\
%		0 	& cos \alpha  & sin \alpha & 0 \\
%		0	& -sin \alpha & cos \alpha & 0 \\
%		0 	& 0 		  & 0          & 1
%	\end{pmatrix}
%\end{equation}
%
%Матрица поворота вокруг оси OY:
%
%\begin{equation}
%	\label{mtr:OY}
%	\begin{pmatrix}
%		cos \alpha 	& 0 & -sin \alpha & 0 \\
%		0 			& 1 & 0 		  & 0 \\
%		sin \alpha	& 0 & cos \alpha  & 0 \\
%		0 			& 0 & 0           & 1
%	\end{pmatrix}
%\end{equation}
%
%Матрица поворота вокруг оси OZ:
%
%\begin{equation}
%	\label{mtr:OZ}
%	\begin{pmatrix}
%		cos \alpha 	 & sin \alpha & 0 & 0 \\
%		-sin \alpha  & cos \alpha & 0 & 0 \\
%		0	 		 & 0		  & 1 & 0 \\
%		0 			 & 0 		  & 0 & 1
%	\end{pmatrix}
%\end{equation}
%
%\clearpage
%
%Перенос в трехмерном пространстве задается значениями смещения положения точки вдоль координатных осей $OX, OY, OZ$ --- $dx, dy, dz$ соответственно.
%
%Матрица переноса имеет вид:
%
%\begin{equation}
%	\label{mtr:move}
%	\begin{pmatrix}
%		1  & 0  & 0  & 0 \\
%		0  & 1  & 0  & 0 \\
%		0  & 0  & 1  & 0 \\
%		dx & dy	& dz & 1
%	\end{pmatrix}
%\end{equation}
%
%Масштабирование задается значениями коэффициентов масштабирования по каждому из направлений $OX, OY, OZ$ --- $kx, ky, kz$.
%
%Матрица масштабирования имеет вид:
%
%\begin{equation}
%	\label{mtr:scale}
%	\begin{pmatrix}
%		k_x & 0   & 0   & 0 \\
%		0   & k_y & 0   & 0 \\
%		0   & 0   & k_z & 0 \\
%		0   & 0	  & 0   & 1
%	\end{pmatrix}
%\end{equation}
%
%\subsection{Модель освещения Ламберта}
%
%Согласно общему алгоритму построения изображения (рисунок \ref{img:generalImageConstrAlg}) третьим шагом является вычисление интенсивности в каждой точке ландшафта с использованием модели освещения Ламберта.
%
%Модель Ламберта основана на законе Ламберта \cite{info_lightLambert}, утверждающем, что интенсивность отраженного света пропорциональна косинусу угла между нормалью поверхности и вектором направления от точки до источника. 
%Это означает, что поверхности, обращенные к источнику света, будут ярче, а те, что обращены в сторону --- менее освещены (см. рисунок \ref{img:design_lambert} для иллюстрации векторов $N$ и $L$).
%
%\clearpage
%
%\includeimage
%{design_lambert} % Имя файла без расширения (файл должен быть расположен в директории inc/img/)
%{f} % Обтекание (без обтекания)
%{h} % Положение рисунка (см. figure из пакета float)
%{0.4\textwidth} % Ширина рисунка
%{Пример расположения векторов $N$ и $L$ в модели освещения Ламберта \cite{info_lightModels}} % Подпись рисунка
%
%
%Пусть:
%\begin{itemize}[label*=--]
%	\item $\vec{L}$ --- единичный вектор направления от точки до источника;
%	\item $\vec{N}$ --- единичный вектор нормали;
%	\item $I$ --- результирующая интенсивность света в точке;
%	\item $I_0$ --- интенсивность источника света;
%	\item $K_d$ --- коэффициент диффузного освещения.
%\end{itemize}
%
%Формула расчета интенсивности имеет следующий вид:
%
%\begin{equation}
%	\label{equ:lambert}
%	I = I_0 \cdot K_d \cdot cos(\vec{L}, \vec{N}) \cdot I_d = I_0 \cdot K_d \cdot (\vec{L}, \vec{N})
%\end{equation}
%
%
%\subsection{Алгоритм, использующий Z-буфер}
%
%На рисунке \ref{img:zbufferAlg} представлена схема алгоритма, использующего Z-буфер.
%
%\includesvgimage
%{zbufferAlg} % Имя файла без расширения (файл должен быть расположен в директории inc/img/)
%{f} % Обтекание (без обтекания)
%{h} % Положение рисунка (см. figure из пакета float)
%{0.8\textwidth} % Ширина рисунка
%{Схема алгоритма, использующего Z-буфер} % Подпись рисунка
%
%\clearpage
%
%\subsection{Алгоритм закраски по Гуро}
%
%На рисунке \ref{img:shadingByGuroAlg} представлена схема алгоритма закраски по Гуро.
%
%\includesvgimage
%{shadingByGuroAlg} % Имя файла без расширения (файл должен быть расположен в директории inc/img/)
%{f} % Обтекание (без обтекания)
%{h} % Положение рисунка (см. figure из пакета float)
%{0.65\textwidth} % Ширина рисунка
%{Схема алгоритма закраски по Гуро} % Подпись рисунка
%
%\clearpage
%
%\subsection{Алгоритм, использующий Z-буфер, объединенный с закраской по Гуро}
%
%На рисунке \ref{img:zbufferWithShadingByGuroAlg} представлена схема алгоритма, использующего Z-буфер, объединенного с закраской по Гуро.
%
%\includesvgimage
%{zbufferWithShadingByGuroAlg} % Имя файла без расширения (файл должен быть расположен в директории inc/img/)
%{f} % Обтекание (без обтекания)
%{h} % Положение рисунка (см. figure из пакета float)
%{0.6\textwidth} % Ширина рисунка
%{Схема алгоритма, использующего Z-буфер, объединенного с закраской по Гуро} % Подпись рисунка
%
%\subsection{Выбор типов и структур данных}
%
%Для реализации разрабатываемого ПО необходимо использование структур данных, представленных в таблице \ref{tbl:tableDataPresent}.
%
%\begin{table}[ht]
%	\begin{center}
%		\begin{threeparttable}
%			\caption{Представление данных в ПО}
%			\label{tbl:tableDataPresent}
%			\begin{tabular}{|c|c|}
%				\hline
%				\bfseries Данные & \bfseries Представление  \\
%				\hline
%				\makecell{Точка в \\ трехмерном пространстве}  & Координаты по осям $x$, $y$ и $z$  \\
%				\hline
%				\makecell{Вектор в \\ трехмерном пространстве} & \makecell{Точка в \\ трехмерном пространстве}  \\
%				\hline
%				Карта высот & \makecell{Двумерный массив точек в  \\ трехмерном пространстве} \\
%				\hline
%				\makecell{Плоскость}  &  \makecell{
%					Три точки в \\ трехмерном пространстве  \\ cо  значениями  \\ коэффициентов плоскости}		\\		
%				\hline
%				\makecell{Источник света}  &  \makecell{Точка трехмерного пространства  \\ со значениями \\ интенсивности источника света и \\ коэффициента диффузного освещения} \\
%				\hline
%				Ландшафт & \makecell{Карта высот,  \\ длина и ширина карты высот} \\
%				\hline
%			\end{tabular}
%		\end{threeparttable}
%	\end{center}
%\end{table}
%
%\subsection{Описание структуры программного обеспечения}
%
%На рисунке \ref{img:UMLClassStructure} представлена диаграмма классов разрабатываемого программного обеспечения.
%
%\includesvgimage
%{UMLClassStructure} % Имя файла без расширения (файл должен быть расположен в директории inc/img/)Bmatrix
%{f} % Обтекание (без обтекания)
%{h} % Положение рисунка (см. figure из пакета float)
%{0.92\textwidth} % Ширина рисунка
%{Диаграмма классов разрабатываемого ПО} % Подпись рисунка
%
%\clearpage
%
%В разрабатываемом программном обеспечении реализуются следующие классы: 
%
%\begin{itemize}[label*=--]
%	\item \texttt{PerlinNoise} --- класс, хранящий параметры алгоритма шума Перлина и реализующий возможность генерации высоты для переданной точки;
%	\item \texttt{Plane} --- класс для представления плоскости, являющейся треугольным полигоном;
%	\item \texttt{Transform} --- класс для осуществления афинных преобразований;
%	\item \texttt{Light} --- класс  для представления точечного источник света;
%	\item \texttt{LightManager} --- класс для вычисления интенсивностей света в точке;
%	\item \texttt{Landscape} --- класс для представления трехмерного ландшафта;
%	\item \texttt{LandscapeManager} --- класс для осуществления всех операций по изменению ландшафта;
%	\item \texttt{Renderer} --- класс для растеризации ландшафта и вывода его на экран.
%\end{itemize}
%
\section*{Вывод}

В данном разделе была представлена диаграмма проектируемой базы данных, были описаны ее сущности, а также были описаны требуемые ограничения целостности. Была представлена схема алгоритма работы триггера на обновление среднего рейтинга продаваемого товара при добавлении/удалении отзыва, а также схема алгоритма функции для получения среднего рейтинга продаваемого товара.
