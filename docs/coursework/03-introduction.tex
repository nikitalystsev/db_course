
\chapter*{ВВЕДЕНИЕ}
\addcontentsline{toc}{chapter}{ВВЕДЕНИЕ}

Клиенты всегда в поиске выгодных предложений \cite{info_theme_cmp_prices}. Современный рынок предлагает широкий спектр продовольственных товаров, и среди этого многообразия бывает сложно определить оптимальное предложение по цене. Для решения этой проблемы создаются сервисы сравнения цен, которые позволяют удобным образом сравнивать цены на товары в различных магазинах \cite{info_cmp_prices}. 

На сегодняшний день пользователи все чаще используют сервисы сравнения цен на товары \cite{info_stat_cmp_prices}. Эти платформы не только помогают сэкономить деньги, но и позволяют находить альтернативные продукты, которые могут быть более качественными или полезными. В результате, такие сервисы становятся важным инструментом для осознанного потребления и рационального выбора.

Целью данного курсового проекта является разработка базы данных для сравнительного анализа цен на элементы продуктовой корзины первой необходимости.

Для достижения поставленной цели необходимо решить следующие задачи: 

\begin{enumerate}[label={\arabic*)}]
	\item провести анализ предметной области;
	\item проанализировать существующие решения;
	\item спроектировать базу данных, описать ее сущности и связи между сущностями;
	\item выбрать подходящие средства реализации;
	\item реализовать базу данных;
	\item провести исследования созданной базы данных.
\end{enumerate}

