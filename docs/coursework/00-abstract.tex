\begin{essay}{}
	Ключевые слова: базы данных, продуктовая корзина, акции на товары, сертификаты на товары, сравнение цены на товар.
	
	Объектом разработки является база данных.
	
	Цель работы -- разработка базы данных для сравнительного анализа цен на элементы продуктовой корзины первой необходимости.
	
	В аналитической части (\ref{analysis}) был проведен анализ предметной области, были рассмотрены существующие решения, формализованы роли пользователей и описаны возможные действия для каждой роли. Также была создана ER-диаграмма проектируемой базы данных и выбрана реляционная СУБД исходя из их классификации по модели данных.
	
	В конструкторской  части (\ref{design}) была представленая ER-модель разрабатываемой базы данных, описаны сущности и ограничения целостности. Были разработаны after триггер для пересчета среднего рейтинга продаваемого товара и функция для получения среднего рейтинга продаваемого товара.
	
	В технологической части (\ref{implementation}) были выбраны СУБД PostgreSQL для реализации базы данных, язык Golang для написания приложения для взаимодействия с базой данных, среда разработки. Были приведены листинги команд для создания базы данных, таблиц, ограничений целостности. Также был описано консольный интерфейс приложения для взаимодействия с базой данных.
	
	В исследовательской части (\ref{research}) было проведено исследование зависимости среднего времени ответа от числа запросов в секунду с использованием кеширования приложения и без использования кеширования приложения. Результаты исследования показали, что использование кеширования приложения увеличивает число обрабатываемых запросов в секунду в  $\approx$ 2.46 и уменьшает среднее время ответа $\approx$ 1.71 раз, а использование индекса с ростом числа записей в таблице сокращает время выполнения запроса $\approx$ в 2.77 раза.
\end{essay}