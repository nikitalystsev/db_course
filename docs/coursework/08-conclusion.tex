\chapter*{ЗАКЛЮЧЕНИЕ}
\addcontentsline{toc}{chapter}{ЗАКЛЮЧЕНИЕ}

В ходе выполнения курсового проекта были решены следующие задачи: 

\begin{enumerate}[label={\arabic*)}]
	\item проведен анализ предметной области;
	\item проанализированы существующие решения;
	\item спроектирована базу данных, описать ее сущности и связи между сущностями;
	\item выбраны подходящие средства реализации;
	\item реализована база данных;
	\item проведено исследования созданной базы данных.
\end{enumerate}

Цель работы, а именно разработка базы данных для сравнительного анализа цен на элементы продуктовой корзины первой необходимости, также была достигнута.

Добавление кеширования на уровне приложения позволило значительно увеличить число обрабатываемых запросов в секунду (с использованием кеша максимальное число обрабатываемых запросов в секунду увеличилось с 473 до 1165, что в $\approx$ 2.46 раза больше, чем без использования кеша) а также сократить время ответа на запросы практически в 2 раза (максимальное среднее время ответа без кеша равно 607.249, а максимальное время ответа с кешем равно 354.096 мс, что меньше $\approx$ 1.71 раза).

Использование индекса позволило значительно сократить время выполнения запроса (при 10 записях в таблице разница во времени выполнения была всего 0.3 мкс, в то время как при 1000 записей в таблице запрос с использованием индекса выполняется быстрее $\approx$ в 2.77 раза).